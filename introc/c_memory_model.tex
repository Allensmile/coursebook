\section{C Memory Model}

The C memory model is probably unlike most that you've seen before. Instead of allocating an object with type safety, we either use an automatic variable or request a sequence of bytes with \keyword{malloc} or another family member and later we \keyword{free} it.

\subsection{Structs}

In low-level terms, a struct is just a piece of contiguous memory, nothing more.
Just like an array, a struct has enough space to keep all of its members.
But unlike an array, it can store different types. Consider the contact struct declared above

\begin{lstlisting}[language=C]
struct contact {
    char firstname[20];
    char lastname[20];
    unsigned int phone;
};

struct contact bhuvan;
\end{lstlisting}

\begin{lstlisting}[language=C]
/* a lot of times we will do the following typedef
 so we can just write contact contact1 */

typedef struct contact contact;
contact bhuvan;

/* You can also declare the struct like this to get
 it done in one statement */
typedef struct optional_name {
    ...
} contact;
\end{lstlisting}

If you compile the code without any optimizations and reordering, you can expect the addresses of each of the variables to look like this.

\begin{lstlisting}[language=C]
&bhuvan           // 0x100
&bhuvan.firstname // 0x100 = 0x100+0x00
&bhuvan.lastname  // 0x114 = 0x100+0x14
&bhuvan.phone     // 0x128 = 0x100+0x28
\end{lstlisting}

Because all your compiler does is say `reserve this much space, and I will go and calculate the offsets of whatever variables you want to write to'. The offsets are where the variable starts at. The phone variables starts at the \keyword{0x128}th bytes and continues for sizeof(int) bytes, but not always. \textbf{Offsets don't determine where the variable ends though}. Consider the following hack that you see in a lot of kernel code.

\begin{lstlisting}[language=C]

typedef struct {
    int length;
    char c_str[0];
} string;

const char* to_convert = "bhuvan";
int length = strlen(to_convert);

// Let's convert to a c string
string* bhuvan_name;
bhuvan_name = malloc(sizeof(string) + length+1);
/*
Currently, our memory looks like this with junk in those black spaces
                ___ ___ ___ ___ ___ ___ ___ ___ ___ ___ ___
 bhuvan_name = |___|___|___|___|___|___|___|___|___|___|___|

*/


bhuvan_name->length = length;
/*
This writes the following values to the first four bytes
The rest is still garbage
                ___ ___ ___ ___ ___ ___ ___ ___ ___ ___ ___
 bhuvan_name = | 0 | 0 | 0 | 6 |___|___|___|___|___|___|___|

*/


strcpy(bhuvan_name->c_str, to_convert);
/*
Now our string is filled in correctly at the end of the struct

                ___ ___ ___ ___ ___ ___ ___ ___ ___ ___ ____
 bhuvan_name = | 0 | 0 | 0 | 6 | b | h | u | v | a | n | \0 |
                                                           ‾
*/

strcmp(bhuvan_name->c_str, "bhuvan") == 0 //The strings are equal!
\end{lstlisting}

What that zero length array does is point to the \textbf{end of the struct} this means that the compiler will leave room for all of the elements calculated with respect to their size on the operating system (ints, chars, etc). The zero length array will take up no bytes of space.
Since structs are just continuous pieces of memory, we can allocate \textbf{more} space than required and use the extra space as a place to store extra bytes.
Although this seems like a parlor trick, it is an important optimization because to have a variable length string any other way, one would need to have two different memory allocation calls.
This is highly inefficient for doing something as common in programming as string manipulation.

\subsubsection{Extra: Struct packing}

Structs may require something called \href{http://www.catb.org/esr/structure-packing/}{padding} (tutorial).
\textbf{We do not expect you to pack structs in this course, just know that it is there} This is because in the early days (and even now) when you have to an address from memory you have to do it in 32bit or 64bit blocks.
This also meant that you could only request addresses that were multiples of that.
Meaning that

\begin{lstlisting}[language=C]
struct picture{
    int height;
    pixel** data;
    int width;
    char* encoding;
}
\end{lstlisting}
You think picture looks like this. One box is four bytes

\begin{verbatim}
| h |  data | w | encod |
|---+-------+---+-------|
|___|___|___|___|___|___|
\end{verbatim}

With struct packing, would conceptually look like this

\begin{lstlisting}[language=C]
struct picture{
    int height;
    char slop1[4];
    pixel** data;
    int width;
    char slop2[4];
    char* encoding;
}
\end{lstlisting}

Here is some more ASCII art for you

\begin{verbatim}
| h | ? | data  | w | ? | encod |
|---+---+-------+---+---+-------|
|___|___|___ ___|___|___|___ ___|
\end{verbatim}

This is on a 64-bit system. This is not always the case because
sometimes your processor supports unaligned accesses. What does this mean? Well there are two options you can set an attribute

\begin{lstlisting}[language=C]
struct __attribute__((packed, aligned(4))) picture{
    int height;
    pixel** data;
    int width;
    char* encoding;
}
\end{lstlisting}

\begin{verbatim}
| h |  data | w | encod |
|---+-------+---+-------|
|___|___|___|___|___|___|
\end{verbatim}

But now, every time I want to access \keyword{data} or \keyword{encoding},
I have to do two memory accesses.
The other thing you can do is reorder
the struct, although this is not always possible

\begin{lstlisting}[language=C]
struct picture{
    int height;
    int width;
    pixel** data;
    char* encoding;
}
\end{lstlisting}

\begin{verbatim}
| h | w | data  | encod |
|---+---+---+---+-------|
|___|___|___|___|___|___|
\end{verbatim}

\subsection{Strings in C}

In C we have
\href{https://en.wikipedia.org/wiki/Null-terminated_string}{Null
	Terminated} strings rather than
\href{https://en.wikipedia.org/wiki/String_(computer_science)\#Length-prefixed}{Length
	Prefixed} for historical reasons. What that means for your average everyday programming is that you need to remember the null character!
A string in C is defined as a bunch of bytes until you reach `\0' or the NUL Byte.

\subsection{Places for strings}

Whenever you define a constant string -- one in the form \keyword{char*\ str\ =\ "constant"} -- that string is stored in the \emph{data} depending on your architecture which is \textbf{read-only} meaning that any attempt to modify the string will cause a SEGFAULT.
One can also declare strings to be either in the writable data segment or the stack. To do so, just specify a length for the string or put brackets instead of a pointer \keyword{char str[] = "mutable"} and put in the global scope or the function scope for the data segment or the stack respectively.
If one, however, \keyword{malloc}'s space, one can change that string to be whatever they want.
Forgetting to NUL terminate a string is a big effect on the strings! Bounds checking is important.
The heartbleed bug mentioned earlier in the book is partially because of this.

Strings in C are represented as characters in memory.
The end of the string includes a NUL (0) byte.
So "ABC" requires four(4) bytes.
The only way to find out the length of a C string is to keep reading memory until you find the NULL byte.
C characters are always exactly one byte each.

\subsubsection{String constants are constant}

A string constant is naturally constant.
Any write will cause the operating system to produce a SEGFAULT.

\begin{lstlisting}[language=C]
char array[] = "Hi!"; // array contains a mutable copy
strcpy(array, "OK");

char *ptr = "Can't change me"; // ptr points to some immutable memory
strcpy(ptr, "Will not work");
\end{lstlisting}

String literals are character arrays stored in the code segment of the program, which is immutable.
Two string literals may share the same space in memory.
An example follows.

\begin{lstlisting}[language=C]
char *str1 = "Bhuvy likes books";
char *str2 = "Bhuvy likes books";
\end{lstlisting}

The strings pointed to by \keyword{str1} and \keyword{str2} may actually reside in the same location in memory.

Char arrays, however, contain the literal value which has been copied from the code segment into either the stack or static memory.
These following char arrays do not reside in the same place in memory.

\begin{lstlisting}[language=C]
char arr1[] = "Bhuvy also likes to write";
char arr2[] = "Bhuvy also likes to write";
\end{lstlisting}

Here are some common ways to initialize a string include. Where do they reside in memory?

\begin{lstlisting}[language=C]
char *str = "ABC";
char str[] = "ABC";
char str[]={'A','B','C','\0'};
\end{lstlisting}

\begin{lstlisting}[language=C]
char ary[] = "Hello";
char *ptr = "Hello";
\end{lstlisting}

We can also print out the pointer and the contents of a c string very easily. Here is some boilerplate code to do the printout.

\begin{lstlisting}[language=C]
char ary[] = "Hello";
char *ptr = "Hello";
// Print out address and contents
printf("%p : %s\n", ary, ary);
printf("%p : %s\n", ptr, ptr);
\end{lstlisting}

As mentioned before, the char array is mutable, so we can change its contents.
Be careful not to write bytes beyond the end of the array though.
Also again be careful not to get the two mixed up.

\begin{lstlisting}[language=C]
strcpy(ary, "World"); // OK
strcpy(ptr, "World"); // NOT OK - Segmentation fault (crashes)
\end{lstlisting}

We can, however, unlike the array, we change \keyword{ptr} to point to another piece of memory,

\begin{lstlisting}[language=C]
ptr = "World"; // OK!
ptr = ary; // OK!
ary = "World"; // NO won't compile
// ary is doomed to always refer to the original array.
printf("%p : %s\n", ptr, ptr);
strcpy(ptr, "World"); // OK because now ptr is pointing to mutable memory (the array)
\end{lstlisting}

What to take away from this is that pointers * can point to any type of memory while the char arrays mentioned earlier will always refer to the same piece of memory.
In a more common case, pointers will point to heap memory in which case the memory referred to by the pointer \textbf{can} be modified.


