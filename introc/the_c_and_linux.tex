\section{The C and Linux}

So up until this point we've covered C's language fundamentals.
We'll now be focusing our attention to C and the POSIX variety of functions available to us to interact with the operating systems.
We will talk about portable functions, for example \keyword{fwrite} \keyword{printf}.
We will be evaluating the internals and scrutinizing them under the POSIX models and more specifically GNU/Linux.
There are a number of things to that philosophy that makes the rest of this easier to know, so we'll put those things here.

\subsection{Everything is a file}

One POSIX mantra is that everything is a file.
Although that has become recently outdated, and moreover wrong, it is the convention we still use today.
What this statement means is that everything is a file descriptor, which is an integer. For example, here is a file object, a network socket, and a kernel object. These are all references to records in the kernel's file descriptor table.

\begin{lstlisting}[language=C]
int file_fd = open(...);
int network_fd = socket(...);
int kernel_fd = epoll_create1(...);
\end{lstlisting}

And operations on those objects are done through system calls.
One last thing to note before we move on is that the file descriptors are merely \textit{pointers}.
Imagine that each of the file descriptors in the example actually refer to an entry in a table of objects that the operating system picks and chooses from (that is, the file descriptor table).
Objects can be allocated and deallocated, closed and opened, etc.
The program interacts with these objects by using the API specified through system calls, and library functions.

\subsection{System Calls}

Before we dive into common C functions, we need to know what a system call is.
If you are a student and have completed HW0, feel free to gloss over this section.

A system call is an operation that the kernel carries out.
First, the operating system prepares a system call. Next, the kernel executes the system call to the best of its ability in kernel space, and is a privileged operation.
In the previous example, we got access to a file descriptor object.
We can now also write some bytes to the file descriptor object that represents a file, and the operating system will do its best to get the bytes written to the disk.

\begin{lstlisting}[language=C]
write(file_fd, "Hello!", 6);
\end{lstlisting}

When we say the kernel tries its best, this includes the possibility that the operation could fail for a number of reasons. Some of them are: the file is no longer valid, the hard drive failed, the system was interrupted etc.
The way that you communicate with the outside system is with system calls.
An important thing to note is that system calls are expensive.
Their cost in terms of time and CPU cycles has recently been decreased, but try to use them as sparingly as possible.

\subsection{C System Calls}

Many C functions that we will discuss in the next sections will be an abstraction that calls the correct underlying system call, based on the current platform. 
Their Windows implementation may be entirely different from that of other operating systems.
Nevertheless, we will be studying these in the context of their Linux implementation.


