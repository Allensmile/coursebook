\section{Common Bugs}

\subsection{Null Bytes}

What's wrong with this code?

\begin{lstlisting}[language=C]
void mystrcpy(char*dest, char* src) {
  // void means no return value
  while( *src ) { dest = src; src ++; dest++; }
}
\end{lstlisting}

In the above code it simply changes the dest pointer to point to source string.
Also the NUL bytes are not copied.
Here is a better version -

\begin{lstlisting}[language=C]
while( *src ) { *dest = *src; src ++; dest++; }
*dest = *src;
\end{lstlisting}

Note that it is also common to see the following kind of implementation, which does everything inside the expression test, including copying the NUL byte.
However, this is not good style, as a result of doing multiple operations in the same line.

\begin{lstlisting}[language=C]
while( (*dest++ = *src++ )) {};
\end{lstlisting}

\subsection{Double Frees}

A double free error is when a program accidentally attempt to free the same allocation twice.

\begin{lstlisting}[language=C]
int *p = malloc(sizeof(int));
free(p);

*p = 123; // Oops! - Dangling pointer! Writing to memory we don't own anymore

free(p); // Oops! - Double free!
\end{lstlisting}

The fix is first to write correct programs!
Secondly, it is a good habit to set pointers to NULL, once the memory has been freed.
This ensures that the pointer cannot be used incorrectly without the program crashing.

\begin{lstlisting}[language=C]
p = NULL; // No dangling pointers
\end{lstlisting}

\subsection{Returning pointers to automatic variables}

\begin{lstlisting}[language=C]
int *f() {
  int result = 42;
  static int imok;
  return &imok; // OK - static variables are not on the stack
  return &result; // Not OK
}
\end{lstlisting}

Automatic variables are bound to stack memory only for the lifetime of the function.
After the function returns, it is an error to continue to use the memory.

\subsection{Insufficient memory allocation}

\begin{lstlisting}[language=C]
struct User {
  char name[100];
};
typedef struct User user_t;

user_t *user = (user_t *) malloc(sizeof(user));
\end{lstlisting}

In the above example, we needed to allocate enough bytes for the struct.
Instead, we allocated enough bytes to hold a pointer.
Once we start using the user pointer we will corrupt memory.
The correct code is shown below.

\begin{lstlisting}[language=C]
struct User {
  char name[100];
};
typedef struct User user_t;

user_t * user = (user_t *) malloc(sizeof(user_t));
\end{lstlisting}

\subsection{Buffer overflow/ underflow}

A famous example: Heart Bleed performed a memcpy into a buffer that was of insufficient size.
A simple example: implement a strcpy and forget to add one to strlen, when determining the size of the memory required.

\begin{lstlisting}[language=C]
#define N (10)
int i = N, array[N];
for( ; i >= 0; i--) array[i] = i;
\end{lstlisting}

C does not check if pointers are valid.
The above example writes into \keyword{array[10]} which is outside the array bounds.
This can cause memory corruption because that memory location is probably being used for something else.
In practice, this can be harder to spot because the overflow/underflow may occur in a library call.
Here is our old friend gets.

\begin{lstlisting}[language=C]
gets(array); // Let's hope the input is shorter than my array!
\end{lstlisting}


\subsection{Strings require strlen(s)+1 bytes}

Every string must have a null byte after the last characters.
To store the string ``Hi'' it takes 3 bytes: [H] [i] [\backslash 0].

\begin{lstlisting}[language=C]
char *strdup(const char *input) {  /* return a copy of 'input' */
  char *copy;
  copy = malloc(sizeof(char*));     /* nope! this allocates space for a pointer, not a string */
  copy = malloc(strlen(input));     /* Almost...but what about the null terminator? */
  copy = malloc(strlen(input) + 1); /* That's right. */
  strcpy(copy, input);   /* strcpy will provide the null terminator */
  return copy;
}
\end{lstlisting}

\subsection{Using uninitialized variables}

\begin{lstlisting}[language=C]
int myfunction() {
  int x;
  int y = x + 2;
  ...
\end{lstlisting}

Automatic variables hold garbage or bit pattern that happened to be in memory or register.
It is an error to assume that it will always be initialized to zero.

\subsection{Assuming Uninitialized memory will be zeroed}

\begin{lstlisting}[language=C]
void myfunct() {
  char array[10];
  char *p = malloc(10);
\end{lstlisting}

Automatic (temporary variables) are not automatically initialized to zero.
Heap allocations using malloc are not automatically initialized to zero.
