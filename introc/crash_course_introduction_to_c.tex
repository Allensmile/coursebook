\section{Crash course introduction to C}

The canonical way to start learning C is by starting with the \emph{hello world} program.
The original example that Kernighan and Ritchie proposed way back when hasn't changed that much.

\begin{lstlisting}[language=C]
#include <stdio.h>
int main(void) {
  printf("Hello World\n");
  return 0;
}
\end{lstlisting}

\begin{enumerate}
	\item The \keyword{\#include} directive takes the file \keyword{stdio.h} (which stands for \textbf{st}an\textbf{d}ard \textbf{i}nput and \textbf{o}utput) located somewhere in your operating system, copies the text, and substitutes it where the \keyword{\#include} was.
	\item The \keyword{int main(void)} is a function declaration.
    The first word \keyword{int} tells the compiler the return type of the function.
    The part before the parenthesis (\keyword{main}) is the function name.
    In C, no two functions can have the same name in a single compiled program, although shared libraries may be able.
    Then, the parameter list comes after.
    When we provide the parameter list for regular functions \keyword{(void)} that means that the compiler should produce an error if the function is called with a non-zero number of arguments.
    For regular functions having a declaration like \keyword{void func()} means that you are allowed to call the function in a manner similar to \keyword{func(1, 2, 3)}, because there is no delimiter.
    \keyword{main} is a special function.
    There are many ways of declaring \keyword{main} but the ones that you will likely be familiar with are \keyword{int main(void)}, \keyword{int main()}, and \keyword{int main(int argc, char *argv[])}.
	\item \keyword{printf("Hello World\n");} is what we call a function call.
    \keyword{printf} is defined as a part of \keyword{stdio.h}.
    The function has been compiled and lives somewhere else on our machine - the location of the C standard library.
    Just remember to include the header and call the function with the appropriate parameters (a string literal \keyword{"Hello World\n"}).
    If the newline isn't included, the buffer will not be flushed (i.e. the write will not complete immediately).
    It is by convention that buffered IO is not flushed until a newline.
	\item \keyword{return 0}.
    \keyword{main} has to return an integer.
    By convention, \keyword{return 0} means success and anything else means failure.
    Here are some exit codes / statuses with special meaning: \link{http://tldp.org/LDP/abs/html/exitcodes.html}.
\end{enumerate}

\begin{lstlisting}[language=bash]
$ gcc main.c -o main
$ ./main
Hello World
$
\end{lstlisting}

\begin{enumerate}
\item \keyword{gcc} is short for the GNU Compiler Collection which has a host of compilers ready for use.
  The compiler infers from the extension that you are trying to compile a .c file.
\item \keyword{./main} tells your shell to execute the program in the current directory called main.
  The program then prints out "hello world".
\end{enumerate}

If systems programming was as easy as writing hello world though, our jobs would be much easier.

\subsection{Preprocessor}

What is the preprocessor?
Preprocessing is a copy and paste operation that the compiler performs \textbf{before} actually compiling the program.
The following is an example of substitution

\begin{lstlisting}[language=C]
// Before preprocessing
#define MAX_LENGTH 10
char buffer[MAX_LENGTH]

// After preprocessing
char buffer[10]
\end{lstlisting}

There are side effects to the preprocessor though.
One problem is that the preprocessor needs to be able to tokenize properly, meaning trying to redefine the internals of the C language with a preprocessor may be impossible.
Another problem is that they can't be nested infinitely - there is a bounded depth where they need to stop.
Macros are also just simple text substitutions, without semantics.
For example, look at what can happen if a macro tries to perform an inline modification.

\begin{lstlisting}[language=C]
#define min(a,b) ((a)<(b) ? (a) : (b))
int main() {
  int x = 4;
  if(min(x++, 5)) printf("%d is six", x);
  return 0;
}
\end{lstlisting}

Macros are simple text substitution so the above example expands to \keyword{x++\ \textless{}\ 100\ ?\ x++\ :\ 100} (parenthesis omitted for clarity).
In this case, it is opaque what gets printed out, but it will be 6.
Can you try to figure out why?
Also, consider the edge case when operator precedence comes into play.

\begin{lstlisting}[language=C]
#define min(a,b) a < b ? a : b
int x = 99;
int r = 10 + min(99, 100); // r is 100!
// This is what it is expanded to
int r = 10 + 99 < 100 ? 99 : 100
// Which means
int r = (10 + 99) < 100 ? 99 : 100
\end{lstlisting}

You can also have logical problems with the flexibility of certain parameters.
One common source of confusion is with static arrays and the \keyword{sizeof} operator.

\begin{lstlisting}[language=C]
#define ARRAY_LENGTH(A) (sizeof((A)) / sizeof((A)[0]))
int static_array[10]; // ARRAY_LENGTH(static_array) = 10
int* dynamic_array = malloc(10); // ARRAY_LENGTH(dynamic_array) = 2 or 1 consistently
\end{lstlisting}

What is wrong with the macro?
Well, it works if a static array is passed in because \keyword{sizeof} a static array returns the number of bytes that array takes up, and dividing it by the \keyword{sizeof(an\_element)} would give you the number of entries.
But if passed a pointer to a piece of memory, taking the sizeof the pointer and dividing it by the size of the first entry won't always give us the size of the array.

\subsection{Extra: Includes and conditionals}

The other preprocessor include is the \keyword{\#include} directive and conditionals.
The include directive is explained by example.

\begin{lstlisting}[language=C]
// foo.h
int bar();
\end{lstlisting}

This is our file \keyword{bar.c} unpreprocessed.

\begin{lstlisting}[language=C]
#include "foo.h"
int bar() {
}
\end{lstlisting}

After preprocessing, the compiler sees this

\begin{lstlisting}[language=C]
// foo.c unpreprocessed
int bar();

int bar() {

}
\end{lstlisting}

The other tool is preprocessor conditionals.
If a macro is defined or equal to zeros, that branch is not taken.

\begin{lstlisting}[language=C]
int main() {
  #ifdef __GNUC__
  return 1;
  #else
  return 0;
  #endif
}
\end{lstlisting}

Using \keyword{gcc} your compiler would preprocess the source to the following.

\begin{lstlisting}[language=C]
int main() {
  return 1;
}
\end{lstlisting}

Using \keyword{clang} your compiler would preprocess to this.

\begin{lstlisting}[language=C]
int main() {
  return 0;
}
\end{lstlisting}

