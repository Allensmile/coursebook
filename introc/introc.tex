\chapter{C Programming Language}

\epigraph{If you want to teach systems, don't drum up the programmers, sort the issues, and make PRs. Instead, teach them to yearn for the vast and endless C.}{Antoine de Saint-Exup\'{e}ry (Kinda)}

C is the de-facto programming language to do serious system serious programming. Why? Most kernels are written in largely in C. The Linux kernel \cite{Love} and the XNU kernel \citet{xnukernel} of which Mac OS X is based off. The Windows Kernel uses C++, but doing system programming on that is much harder on windows that UNIX for beginner system programmers. Most of you have some experience with C++, but C is a different beast entirely. You don't have nice abstractions like classes and RAII to clean up memory. You are going to have to do that yourself. C gives you much more of an opportunity to shoot yourself in the foot but lets you do thinks at a much finer grain level.

\section{History of C}

C was developed by Dennis Ritchie and Ken Thompson at Bell Labs back in 1973 \cite{Ritchie:1993:DCL:155360.155580}. Back then, we had gems of programming languages like Fortran, ALGOL, and LISP. The goal of C was two fold. One, to target the most popular computers at the time liek the PDP-7. Two, try and remove some of the lower level constructs like managing registers, programming assembly for jumps and instead create a language that had the power to express programs procedurally (as opposed to mathematically like lisp) with more readable code all while still having the ability to interface with the operating system. It sounded like a tough feat. At first, it was only used internally at Bell Labs along with the UNIX operating system. 

The first "real" standardization is with Brian Kerninghan and Dennis Ritchies book \cite{kernighan1988c}. It is still widely regarded today as the only \gls{portable} set of C instructions. The K\&R book is known as the de-facto standard for learning C.  There were different standards of C from ANSI to ISO after the Unix guides. The one that we will be mainly focusing on is the \gls{POSIX} C library. Now to get the elephant out of the room, the Linux kernel is not entirely POSIX compliant. Mostly, it is because they didn't want to pay the fee for compliance

\todo

\section{Features}

\begin{itemize}
\item Fast. There is nothing separating you and the system.
\item Simple. C and its standard library pose a simple set of portable functions.
\item Memory Management. C let's you manage your memory. This can also bite you if you have memory errors.
\item It's Everywhere. Pretty much every computer that is not embedded has some way of interfacing with C. The standard library is also everywhere. C has stood the test of time as a popular language, and it doesn't look like it is going anywhere.
\end{itemize}

\section{Crash course intro to C}

The only way to start learning C is by starting with hello world. As per the original example that Kernighan and Ritchie proposed way back when, the hello world hasn't changed that much.

\begin{code}[language=C]
#include <stdio.h>
int main(void) { 
    printf("Hello World\n");
    return 0; 
}
\end{code}

\begin{enumerate}
  \item The \keyword{\#include} directive takes the file \keyword{stdio.h} (which stands for \textbf{st}an\textbf{d}ard \textbf{i}nput and \textbf{o}utput) located somewhere in your operating system, copies the text, and substitutes it where the \keyword{\#include} was.
  \item The \keyword{int main(void)} is a function declaration. The first word \keyword{int} tells the compiler what the return type of the function is. The part before the parens (\keyword{main}) is the function name. In C, no two functions can have the same name in a single compiled program, shared libraries are a different touchy subject. Then, what comes after is the paramater list. When we give the parameter list for regular functions \keyword{(void)} that means that the compiler should error if the function is called with any arguments. For regular functions having a declaration like \keyword{void func()} means that you are allowed to call the function like \keyword{func(1, 2, 3)} because there is no delimiter \cite{CITATION_NEEDED}. In the case of \keyword{main}, it is a special function. There are many ways of declaring \keyword{main} but the ones that you will be familiar with are \keyword{int main(void)}, \keyword{int main()}, and \keyword{int main(int argc, char *argv[])}.
  \item \keyword{printf("Hello World\n");} is what we call a function call. \keyword{printf} is defined as a part of \keyword{stdio.h}. The function has been compiled and lives somewhere else on our machine. All we need to do is include the header and call the function with the appropriate parameters (a string literal \keyword{"Hello World\n"}). If you don't have the newline, the buffer will not be flushed. It is by convention that buffered IO is not flushed until a newline. \cite{CITATION_NEEDED}
  \item \keyword{return 0;}. \keyword{main} has to return an integer. By convention, \keyword{return 0} means success and anything else means failure \cite{CITATION_NEEDED}.
\end{enumerate}

\begin{code}
$ gcc main.c -o main
$ ./main
Hello World
$
\end{code}

\begin{enumerate}
\item \keyword{gcc} is short for the GNU-Compiler-Collection which has a host of compilers ready for use. The compiler infers from the extension that you are trying to compile a .c file
\item \keyword{./main} tells your shell to execute the program in the current directory called main. The program then prints out hello world
\end{enumerate}

\section{Preprocessor}

What is the preprocessor? Preprocessing is an operation that the compiler performs \textbf{before} actually compiling the program. It is a copy and paste command. Meaning that if I do the following.

\begin{code}[language=C]
#define MAX_LENGTH 10
char buffer[MAX_LENGTH]
// After
char buffer[10]
\end{code}

But there are side effects to the preprocessor. Does the following code print out.

\begin{code}[language=C]
#define min(a,b) ((a)<(b) ? (a) : (b))
int x = 4;
if(min(x++, 100)) printf("%d is six", x);
\end{code}

Macros are simple text substitution so the above example expands to \keyword{x++\ \textless{}\ 100\ ?\ x++\ :\ 100} (parenthesis omitted for clarity). Now for this case, it will probably still print 6 but consider the edge case when \keyword{x = 99}. Also consider the edge case when operator precedence comes into play.

\begin{code}[language=C]
#define min(a,b) a<b ? a : b
int x = 99;
int r = 10 + min(99, 100); // r is 100!
\end{code}

Macros are simple text substitution so the above example expands to \keyword{10\ +\ 99\ \textless{}\ 100\ ?\ 99\ :\ 100}

\subsection{C Preprocessor logical gotcha}\label{c-preprocessor-logical-gotcha}

\begin{code}[language=C]
#define ARRAY_LENGTH(A) (sizeof((A)) / sizeof((A)[0]))
int static_array[10]; // ARRAY_LENGTH(static_array) = 10
int* dynamic_array = malloc(10); // ARRAY_LENGTH(dynamic_array) = 2 or 1
\end{code}

What is wrong with the macro? Well, it works if we have a static array like the first array because \keyword{sizeof} a static array returns the bytes that array takes up, and dividing it by the \keyword{sizeof(an\_element)} would give you the number of entries. But if we use a pointer to a piece of memory, taking the sizeof the pointer and dividing it by the size of the first entry won't always give us the size of the array.

\section{Language Facilities}

\subsection{Keywords}

C has an assortment of keywords. Here are some constructs that you should know briefly as of C99.

\begin{enumerate}
\item \keyword{break} is a keyword that is used in case statements or looping statements. When used in a case statement, the program jumps to the end of the block.
\\
\begin{code}[language=C]
switch(1) {
  case 1: /* Goes to this switch */
    puts("1");
    break; /* Jumps to the end of the block */
  case 2: /* Ignores this program */
    puts("2");
    break;
} /* Continues here */
\end{code}
\\
In the context of a loop, it breaks out of the inner-most loop. The loop can be either a \keyword{for}, \keyword{while}, or \keyword{do-while} construct
\\
\begin{code}[language=C]
while(1) {
  while(2) {
    break; /* Breaks out of while(2) */
  } /* Jumps here */
  break; /* Breaks out of while(1) */
} /* Continues here */
\end{code}
\item \keyword{const} is a language level construct that tells the compiler that this data should not be modified. If one tries to change a const variable, the program will not even compile. \keyword{const} works a little differently when put before the type, the compiler flips the first type and const. Then the compiler uses a left associativity rule. Meaning that whatever is left of the pointer is constant. This is known as const-correctedness.
\\
\begin{code}[language=C]
const int i = 0; // Same as "int const i = 0"
char *str = ...; // Mutable pointer to a mutable string
const char *const_str = ...; // Mutable pointer to a constant string
char const *const_str2 = ...; // Same as above
const char *const const_ptr_str = ...;
// Constant pointer to a constant string
\end{code}

But, it is important to know that this is a compiler imposed restriction only. There are ways of getting around this and the program will run fine with defined behavior. In systems programming, the only type of memory that you can't write to is system write-protected memory.

\begin{code}[language=C]
const int i = 0; // Same as "int const i = 0"
(*((int *)&i)) = 1; // i == 1 now
const char *ptr = "hi";
*ptr = '\0'; // Will cause a Segmentation Violation
\end{code}

\item \keyword{continue} is a control flow statement that exists only in loop constructions. Continue will skip the rest of the loop body and set the program counter back to the start of the loop before.

\begin{code}[language=C]
int i = 10;
while(i--) {
  if(1) continue; /* This gets triggered */
  *((int *)NULL) = 0;
} /* Then reaches the end of the while loop */
\end{code}

\item \keyword{do \{\} while();} is another loop constructs. These loops execute the body and then check the condition at the bottom of the loop. If the condition is zero, the loop body is not executed and the rest of the program is executed. Otherwise, the loop body is executed.

\begin{code}[language=C]
int i = 1;
do {
  printf("%d\n", i--);
} while (i > 10) /* Only executed once */
\end{code}

\item \keyword{enum}
\item \keyword{extern}
\item \keyword{float}
\item \keyword{for}
\item \keyword{goto}
\item \keyword{if else else-if} are control flow keywords. There are a few ways to use these
\item \keyword{inline}
\item \keyword{register}
\item \keyword{restrict}
\item \keyword{return} is a control flow operator that exits the current function. If the function is \keyword{void} then it simply exits the functions. Otherwise another parameter follows as the return value.

\begin{code}[language=C]

\end{code}
\item \keyword{signed} is a modifier which is rarely used, but it forces an type to be signed instead of unsigned. The reasont that this is so rarely used is because types are signed by default and need to have the \keyword{unsigned} modifier to make them unsigned but it may be useful in cases where you want the compiler to default a signed type like.

\begin{code}[language=C]
int count_bits_and_sign(signed representation) {
  //...
}
\end{code}
\item \keyword{sizeof} is an operator that is evaluated at compile time, which evaluates to the number of bytes that the expression contains. Meaning that when the compiler infers the type the following code changes.
\begin{code}[language=C]
char a = 0;
printf("%zu", sizeof(a++));
\end{code}

\begin{code}[language=C]
char a = 0;
printf("%zu", 1);
\end{code}

Which then the compiler is allowed to operate on further. A note that you must have a complete definition of the type at compile time or else you may get an odd error. Consider the following
\\
\begin{code}[language=C]
// file.c
struct person;

printf("%zu", sizeof(person));

// file2.c

struct person {
  // Declarations
}
\end{code}

This code will not compile because sizeof is not able to compile \keyword{file.c} without knowing the full declaration of the \texttt{person} struct. That is typically why we either put the full declaration in a header file or we abstract the creation and the interaction away so that users cannot access the internals of our struct.

\keyword{sizeof(ary)}: \keyword{ary} is an array. Returns the number of bytes required for the entire array (5 chars + zero byte = 6 bytes) \keyword{sizeof(ptr)}: Same as sizeof(char *). Returns the number bytes required for a pointer (e.g.~4 or 8 for a 32 bit or 64-bit machine) \keyword{sizeof} is a special operator. Really it's something the compiler substitutes in before compiling the program because the size of all types is known at compile time. When you have \keyword{sizeof(char*)} that takes the size of a pointer on your machine (8 bytes for a 64-bit machine and 4 for a 32 bit and so on). When you try \keyword{sizeof(char{[}{]})}, the compiler looks at that and substitutes the number of bytes that the \textbf{entire} array contains because the total size of the array is known at compile time.

\begin{code}[language=C]
char str1[] = "will be 11";
char* str2 = "will be 8";
sizeof(str1) //11 because it is an array
sizeof(str2) //8 because it is a pointer
\end{code}

Be careful, using sizeof for the length of a string!
\item \keyword{static}
\item \keyword{struct}
\item \keyword{switch case default} Switches are essentially glorified jump statements. Meaning that you take either a byte or an integer and the control flow of the program jumps to that location.
\\
\begin{code}[language=C]
switch(/* char or int */) {
  case INT1: puts("1");
  case INT2: puts("2");
  case INT3: puts("3");
}
\end{code}

If we give a value of 2 then
\\
\begin{code}[language=C]
switch(2) {
  case 1: puts("1"); /* Doesn't run this */
  case 2: puts("2"); /* Runs this */
  case 3: puts("3"); /* Also runs this */
}
\end{code}

The break statement
\item \keyword{typedef} declares an alias for a type. Often used with structs to reduce the visual clutter of having to write `struct' as part of the type.

\begin{code}[language=C]
typedef float real; 
real gravity = 10;
// Also typedef gives us an abstraction over the underlying type used. 
// In the future, we only need to change this typedef if we
// wanted our physics library to use doubles instead of floats.

typedef struct link link_t; 
//With structs, include the keyword 'struct' as part of the original types
\end{code}

In this class, we regularly typedef functions. A typedef for a function
can be this for example

\begin{code}[language=C]
typedef int (*comparator)(void*,void*);

int greater_than(void* a, void* b){
    return a > b;
}
comparator gt = greater_than;
\end{code}

This declares a function type comparator that accepts two \keyword{void*} params and returns an integer.

\item \keyword{union}
\item \keyword{unsigned}
\item \keyword{void} is a two folded keyword. When used in terms of function or parameter definition then it means that it returns no value or accepts no parameter specifically. The following declares a function that accepts no parameters and returns nothing.

\begin{code}[language=C]
void foo(void);
\end{code}


The other use of \keyword{void} is when you are defining. A \keyword{void *} pointer is just a memory address. It is specified as an incomplete type meaning that you cannot dereference it but it can be promoted to any time to any other type. Pointer arithmetic with these pointer is undefined behavior.

\begin{code}[langauge=C]
int *array = void_ptr; // No cast needed
\end{code}

\item \keyword{volatile} is a compiler keyword. This means that the compiler should not optimize its value out. Consider the following simple function.
\\
\begin{code}[language=C]
int flag = 1;
pass_flag(&flag);
while(flag) {
    // Do things unrelated to flag
}
\end{code}
\\
The compiler may, since the internals of the while loop have nothing to do with the flag, optimize it to the following even though a function may alter the data.
\\
\begin{code}[language=C]
while(1) {
    // Do things unrelated to flag
}
\end{code}
If you put the volatile keyword then it forces the compiler to keep the variable in and perform that check. This is particularly useful for cases where you are doing multi-process or multi-threading programs so that we can
\item \keyword{while } represents the traditional \keyword{while} loop. There is a condition at the top of the loop. While that condition evaluates to a non-zero value, the loop body will be run.
\end{enumerate}

\subsection{C data types}
\begin{enumerate}
\item \keyword{char} Represents exactly one byte of data. The number of bits in a byte might vary. \keyword{unsigned char} and \keyword{signed char} mean the exact same thing. This must be aligned on a boundary (meaning you cannot use bits in between two addresses). The rest of the types will assume 8 bits in a byte.
\item \keyword{short (short int)} must be at least two bytes. This is aligned on a two byte boundary, meaning that the address must be divisble by two.
\item \keyword{int} must be at least two bytes. Again aligned to a two byte boundary. \href{http://www.open-std.org/jtc1/sc22/wg14/www/docs/n1124.pdf}{Page 34}. On most machines this will be 4 bytes.
\item \keyword{long (long int)} must be at least four bytes, which are aligned to a four byte boundary. On some machines this can be 8 bytes.
\item \keyword{long long} must be at least eight bytes, aligned to an eight byte boundary.
\item \keyword{float} represents an IEEE-754 single percision floating point number tightly specified by \href{http://ieeexplore.ieee.org/document/4610935/}{IEEE}. This will be four bytes aligned to a four byte boundary on most machines.
\item \keyword{double} represents an IEEE-754 double percision floating point number specified by the same standard, which is aligned to the nearest eight byte boundary.
\end{enumerate}

\subsection{Operators}

\section{Common C Functions}

To find more information about any functions, use the man pages. Note the man pages are organized into sections. Section 2 are System calls. Section 3 are C libraries. On the web, Google \keyword{man 7 open}. In the shell, \keyword{man -S2 open} or \keyword{man -S3 printf}

\subsection{Input/Output}

\keyword{printf} is the function with which most people are familiar. The first parameter is a format string that includes placeholders for the data to be printed. Common format specifiers are \keyword{\%s} treat the argument as a c string pointer, keep printing all characters until the NULL-character is reached; \keyword{\%d} print the argument as an integer; \keyword{\%p} print the argument as a memory address.

A simple example is shown below:

\begin{code}[language=C]
char *name = ... ; int score = ...;
printf("Hello %s, your result is %d\n", name, score);
printf("Debug: The string and int are stored at: %p and %p\n", name, &score );
// name already is a char pointer and points to the start of the array. 
// We need "&" to get the address of the int variable
\end{code}

By default, for performance, \keyword{printf} does not actually write
anything out (by calling write) until its buffer is full or a newline is
printed.

\subsection{How else can I print strings and single characters?}

Use \keyword{puts(\ name\ )} and \keyword{putchar(\ c\ )} where name is a
pointer to a C string and c is just a \keyword{char}

\subsection{How do I print to other file
streams?}\label{how-do-i-print-to-other-file-streams}

Use
\keyword{fprintf(\ \_file\_\ ,\ "Hello\ \%s,\ score:\ \%d",\ name,\ score);}
Where \_file\_ is either predefined `stdout' `stderr' or a FILE pointer
that was returned by \keyword{fopen} or \keyword{fdopen}

\subsection{Can I use file descriptors?}\label{can-i-use-file-descriptors}

Yes! Just use \keyword{dprintf(int\ fd,\ char*\ format\_string,\ ...);}
Just remember the stream may be buffered, so you will need to assure
that the data is written to the file descriptor.

\subsection{How do I print data into a C string?}

Use \keyword{sprintf} or better \keyword{snprintf}.

\begin{code}[language=C]
char result[200];
int len = snprintf(result, sizeof(result), "%s:%d", name, score);
\end{code}

snprintf returns the number of characters written excluding the
terminating byte. In the above example, this would be a maximum of 199.

The following code is vulnerable to buffer overflow. It assumes or
trusts that the input line will be no more than 10 characters, including
the terminating byte.

\begin{code}[language=C]
char buf[10];
gets(buf); // Remember the array name means the first byte of the array
\end{code}

\subsection{Should I Use Gets?}

\keyword{gets} is deprecated in C99 standard and has been removed from
the latest C standard (C11). Programs should use \keyword{fgets} or
\keyword{getline} instead.

Where each has the following structure respectively:

\begin{code}[language=C]
char *fgets (char *str, int num, FILE *stream); 

ssize_t getline(char **lineptr, size_t *n, FILE *stream);
\end{code}

Here's a simple, safe way to read a single line. Lines longer than 9
characters will be truncated:

\begin{code}[language=C]
char buffer[10];
char *result = fgets(buffer, sizeof(buffer), stdin);
\end{code}

The result is NULL if there was an error or the end of the file is
reached. Note, unlike \keyword{gets}, \keyword{fgets} copies the newline
into the buffer, which you may want to discard-

\begin{code}[language=C]
if (!result) { return; /* no data - don't read the buffer contents */}

int i = strlen(buffer) - 1;
if (buffer[i] == '\n') 
    buffer[i] = '\0';
\end{code}

One of the advantages of \keyword{getline} is that will automatically
(re-) allocate a buffer on the heap of sufficient size.

\begin{code}[language=C]
// ssize_t getline(char **lineptr, size_t *n, FILE *stream);

 /* set buffer and size to 0; they will be changed by getline */
char *buffer = NULL;
size_t size = 0;

ssize_t chars = getline(&buffer, &size, stdin);

// Discard newline character if it is present,
if (chars > 0 && buffer[chars-1] == '\n') 
    buffer[chars-1] = '\0';

// Read another line.
// The existing buffer will be re-used, or, if necessary,
// It will be `free`'d and a new larger buffer will `malloc`'d
chars = getline(&buffer, &size, stdin);

// Later... don't forget to free the buffer!
free(buffer);
\end{code}

\subsection{Parsing}

\subsection{string.h}

Why are \keyword{memcpy} and \keyword{memmove} both in \keyword{\textless{}string.h\textgreater{}}? Because strings are essentially raw memory with a null byte at the end of them! \keyword{void\ *memcpy(void\ *dest,\ const\ void\ *src,\ size\_t\ n)} moves \keyword{n} bytes starting at \keyword{src} to \keyword{dest}. \textbf{Be careful}, there is undefined behavior when the memory regions overlap. This is one of the classic works on my machine examples because many times valgrind won't be able to pick it up because it will look like it works on your machine. When the autograder hits, fail. Consider the safer version which is.

\keyword{void\ *memmove(void\ *dest,\ const\ void\ *src,\ size\_t\ n)}
does the same thing as above, but if the memory regions overlap then it
is guaranteed that all the bytes will get copied over correctly.

\subsection{stdlib.h}

To allocate space on the heap, use malloc. There's also realloc and calloc. Typically used with sizeof. e.g.~enough space to hold 10 integers

\begin{code}[language=C]
int *space = malloc(sizeof(int) * 10);
\end{code}

\subsection{Conventions/Errno}

\section{System Calls}

\subsection{What is a system call?}

\subsection{The interplay with library functions}

\subsubsection{Does \keyword{printf} call write or does write call \keyword{printf}?}

\keyword{printf} calls \keyword{write}. \keyword{printf} includes an
internal buffer so, to increase performance \keyword{printf} may not call
\keyword{write} everytime you call \keyword{printf}. \keyword{printf} is a
C library function. \keyword{write} is a system call and as we know
system calls are expensive. On the other hand, \keyword{printf} uses a
buffer which suits our needs better at that point

\section{C Memory Model}

\subsection{Structs}

\subsubsection{Struct packing}

Structs may require something called \href{http://www.catb.org/esr/structure-packing/}{padding} (tutorial).
\textbf{We do not expect you to pack structs in this course, just know that it
is there} This is because in the early days (and even now) when you have
to an address from memory you have to do it in 32bit or 64bit blocks.
This also meant that you could only request addresses that were
multiples of that. Meaning that

\begin{code}[language=C]
struct picture{
    int height;
    pixel** data;
    int width;
    char* enconding;
}
// You think picture looks like this. One box is four bytes
| h   data    w   encod |
|___ ___ ___ ___ ___ ___|
|___|___ ___|___|___ ___|
\end{code}

Would conceptually look like this

\begin{code}[language=C]
struct picture{
    int height;
    char slop1[4];
    pixel** data;
    int width;
    char slop2[4];
    char* enconding;
}

| h   ?   data    w   ?   encod |
|___ ___ ___ ___ ___ ___ ___ ___|
|___|___|___ ___|___|___|___ ___|
\end{code}

This is on a 64-bit system. This is not always the case because
sometimes your processor supports unaligned accesses. What does this
mean? Well there are two options you can set an attribute

\begin{code}[language=C]
struct __attribute__((packed, aligned(4))) picture{
    int height;
    pixel** data;
    int width;
    char* enconding;
}
// Will look like this
| h   data    w   encod |
|___ ___ ___ ___ ___ ___|
|___|___ ___|___|___ ___|

\end{code}

But now, every time I want to access \keyword{data} or \keyword{encoding},
I have to do two memory accesses. The other thing you can do is reorder
the struct, although this is not always possible

\begin{code}[language=C]
struct picture{
    int height;
    int width;
    pixel** data;
    char* enconding;
}
// You think picture looks like this
| h   w    data   encod |
|___ ___ ___ ___ ___ ___|
|___|___|___ ___|___ ___|

\end{code}

\subsection{C Null-Terminated Strings}

Strings in C are represented as characters in memory. The end of the string includes a NULL (0) byte \cite{CITATION_NEEDED}. So ``ABC'' requires four(4) bytes \keyword{{[}\textquotesingle{}A\textquotesingle{},\textquotesingle{}B\textquotesingle{},\textquotesingle{}C\textquotesingle{},\textquotesingle{}\textbackslash{}0\textquotesingle{}{]}}.
The only way to find out the length of a C string is to keep reading memory until you find the NULL byte. C characters are always exactly one byte each.

When you write a string literal \keyword{"ABC"} in an expression the string literal evaluates to a char pointer (\keyword{char\ *}), which points to the first byte/char of the string. This means \keyword{ptr} in the example below will hold the memory address of the first character in the string.

\subsubsection{String constants are constant}

\begin{code}[language=C]
char array[] = "Hi!"; // array contains a mutable copy 
strcpy(array, "OK");

char *ptr = "Can't change me"; // ptr points to some immutable memory
strcpy(ptr, "Will not work");
\end{code}

String literals are character arrays stored in the code segment of the program, which is immutable. Two string literals may share the same space in memory. An example follows:

\begin{code}[language=C]
char *str1 = "Brandon Chong is the best TA";
char *str2 = "Brandon Chong is the best TA";
\end{code}

The strings pointed to by \keyword{str1} and \keyword{str2} may actually reside in the same location in memory.

Char arrays, however, contain the literal value which has been copied from the code segment into either the stack or static memory. These following char arrays do not reside in the same place in memory.

\begin{code}[language=C]
char arr1[] = "Brandon Chong didn't write this";
char arr2[] = "Brandon Chong didn't write this";
\end{code}

\begin{code}[language=C]
char *ptr = "ABC"
\end{code}

Some common ways to initialize a string include:

\begin{code}[language=C]
char *str = "ABC";
char str[] = "ABC";
char str[]={'A','B','C','\0'};
\end{code}

\begin{code}[language=C]
char ary[] = "Hello";
char *ptr = "Hello";
\end{code}

Example

The array name points to the first byte of the array. Both \keyword{ary}
and \keyword{ptr} can be printed out:

\begin{code}[language=C]
char ary[] = "Hello";
char *ptr = "Hello";
// Print out address and contents
printf("%p : %s\n", ary, ary);
printf("%p : %s\n", ptr, ptr);
\end{code}

The array is mutable, so we can change its contents. Be careful not to write bytes beyond the end of the array though. Fortunately, "World" is no longer than "Hello"

In this case, the char pointer \keyword{ptr} points to some read-only memory (where the statically allocated string literal is stored), so we cannot change those contents.

\begin{code}[language=C]
strcpy(ary, "World"); // OK
strcpy(ptr, "World"); // NOT OK - Segmentation fault (crashes)
\end{code}

We can, however, unlike the array, we change \keyword{ptr} to point to
another piece of memory,

\begin{code}[language=C]
ptr = "World"; // OK!
ptr = ary; // OK!
ary = (..anything..) ; // WONT COMPILE
// ary is doomed to always refer to the original array.
printf("%p : %s\n", ptr, ptr);
strcpy(ptr, "World"); // OK because now ptr is pointing to mutable memory (the array)
\end{code}

What to take away from this is that pointers * can point to any type of memory while C arrays {[}{]} can only point to memory on the stack. In a more common case, pointers will point to heap memory in which case the memory referred to by the pointer CAN be modified.

\subsection{Errors}

\subsubsection{Double Frees}

A double free error is when you accidentally attempt to free the same allocation twice.

\begin{code}[language=C]
int *p = malloc(sizeof(int));
free(p);

*p = 123; // Oops! - Dangling pointer! Writing to memory we don't own anymore

free(p); // Oops! - Double free!
\end{code}

The fix is first to write correct programs! Secondly, it's good programming hygiene to reset pointers once the memory has been freed. This ensures the pointer can't be used incorrectly without the program crashing.

Fix:

\begin{code}[language=C]
p = NULL; // Now you can't use this pointer by mistake
\end{code}

\subsubsection{Buffer Overflows}

TODO

Famous example: Heart Bleed (performed a memcpy into a buffer that was of insufficient size). Simple example: implement a strcpy and forget to add one to strlen, when determining the size of the memory required.

\subsubsection{Returning pointers to automatic variables}

\begin{code}[language=C]
int *f() {
    int result = 42;
    static int imok;
    return &imok; // OK - static variables are not on the stack
    return &result; // Not OK
}
\end{code}

Automatic variables are bound to stack memory only for the lifetime of
the function. After the function returns it is an error to continue to
use the memory. \#\# Insufficient memory allocation

\begin{code}[language=C]
struct User {
   char name[100];
};
typedef struct User user_t;

user_t *user = (user_t *) malloc(sizeof(user));
\end{code}

In the above example, we needed to allocate enough bytes for the struct.
Instead, we allocated enough bytes to hold a pointer. Once we start
using the user pointer we will corrupt memory. The correct code is shown
below.

\begin{code}[language=C]
struct User {
   char name[100];
};
typedef struct User user_t;

user_t * user = (user_t *) malloc(sizeof(user_t));
\end{code}


\section{Pointers}

\subsection{Pointer Basics}

\subsubsection{Declaring a Pointer}

A pointer refers to a memory address. The type of the pointer is useful
- it tells the compiler how many bytes need to be read/written. You can
declare a pointer as follows.

\begin{code}[language=C]
int *ptr1;
char *ptr2;
\end{code}

Due to C's grammar, an \keyword{int*} or any pointer is not actually its
own type. You have to precede each pointer variable with an asterisk. As
a common gotcha, the following

\begin{code}[language=C]
int* ptr3, ptr4;
\end{code}

Will only declare \keyword{*ptr3} as a pointer. \keyword{ptr4} will
actually be a regular int variable. To fix this declaration, keep the
\keyword{*} preceding to the pointer

\begin{code}[language=C]
int *ptr3, *ptr4;
\end{code}

Keep this in mind for structs as well. If one does not typedef them, then the pointer goes after the type.

\begin{code}[language=C]
struct person *ptr3;
\end{code}

\subsubsection{Reading/Writing with pointers}

Let's say that we declare a pointer \keyword{int\ *ptr}. For the sake of
discussion, let's say that \keyword{ptr} points to memory address
\keyword{0x1000}. If we want to write to a pointer, we can dereference
and assign \keyword{*ptr}.

\begin{code}[language=C]
*ptr = 0; // Writes some memory.
\end{code}

What C will do is take the type of the pointer which is an \keyword{int}
and writes \keyword{sizeof(int)} bytes from the start of the pointer,
meaning that bytes \keyword{0x1000}, \keyword{0x1001}, \keyword{0x1002},
\keyword{0x1003} will all be zero. The number of bytes written depends on
the pointer type. It is the same for all primitive types but structs are
a little different.

\subsection{Pointer Arithmetic}

You can add an integer to a pointer. However, the pointer type is used
to determine how much to increment the pointer. For char pointers this
is trivial because characters are always one byte:

\begin{code}[language=C]
char *ptr = "Hello"; // ptr holds the memory location of 'H'
ptr += 2; //ptr now points to the first'l'
\end{code}

If an int is 4 bytes then ptr+1 points to 4 bytes after whatever ptr is
pointing at.

\begin{code}[language=C]
char *ptr = "ABCDEFGH";
int *bna = (int *) ptr;
bna +=1; // Would cause iterate by one integer space (i.e 4 bytes on some systems)
ptr = (char *) bna;
printf("%s", ptr);
/* Notice how only 'EFGH' is printed. Why is that? Well as mentioned above, when performing 'bna+=1' we are increasing the **integer** pointer by 1, (translates to 4 bytes on most systems) which is equivalent to 4 characters (each character is only 1 byte)*/
return 0;
\end{code}

Because pointer arithmetic in C is always automatically scaled by the
size of the type that is pointed to, you can't perform pointer
arithmetic on void pointers.

You can think of pointer arithmetic in C as essentially doing the
following

If I want to do

\begin{code}[language=C]
int *ptr1 = ...;
int *offset = ptr1 + 4;
\end{code}

Think

\begin{code}[language=C]
int *ptr1 = ...;
char *temp_ptr1 = (char*) ptr1;
int *offset = (int*)(temp_ptr1 + sizeof(int)*4);
\end{code}

To get the value. \textbf{Every time you do pointer arithmetic, take a
deep breath and make sure that you are shifting over the number of bytes
you think you are shifting over.}

\subsection{What is a void pointer?}

A pointer without a type (very similar to a void variable). Void pointers are used when either a datatype you're dealing with is unknown or when you're interfacing C code with other programming languages. You can think of this as a raw pointer, or just a memory address. You cannot directly read or write to it because the void type does not have a size. For Example

\begin{code}[language=C]
void *give_me_space = malloc(10);
char *string = give_me_space;
\end{code}

This does not require a cast because C automatically promotes \keyword{void*} to its appropriate type. \textbf{Note:}

\keyword{gcc} and \keyword{clang} are not total ISO-C compliant, meaning that they will let you do arithmetic on a void pointer. They will treat it as a \keyword{char *} pointer. Do not do this because it may not work with all compilers!

\section{Shell}

What do you actually use to run your program? A shell! A shell is a programming language that is running inside your terminal. A terminal is merely a window to input commands. Now, on POSIX we usually have one shell called \keyword{sh} that is linked to a POSIX compliant shell called \keyword{dash}. Most of the time, you use a shell called \keyword{bash} that is not entirely POSIX compliant but has some nifty built in features. If you want to be even more advanced, \keyword{zsh} has some more powerful features like tab complete on programs and fuzzy patterns, but that is neither here nor there. 

\subsection{Common Utilities}

\begin{enumerate}
\item \keyword{cat}
\item \keyword{diff}
\item \keyword{grep}
\item \keyword{ls}
\item \keyword{cd}
\item \keyword{man}
\item \keyword{make}
\end{enumerate}

\subsection{Syntactic}

Shells have many useful utilities like saving output to a file using redirection \keyword{>}. This overwrites the file from the beginning. If you only meant to append to the file, you can use \keyword{>>}. Unix also allows file descriptor swapping. This means that you can take the output going to one file descriptor and make it seem like its coming out of another. The most common one is \keyword{2>&1} which means take the stderr and make it seem like it is coming out of standard out. This is important because when you use \keyword{>} and \keyword{>>} they only write the standard output of the file. There are some examples below.

\begin{code}[language=console]
$ ./program > output.txt # To overwrite
$ ./program >> output.txt # To append
$ ./program 2>&1 > output_all.txt # stderr & stdout
$ ./program 2>&1 > /dev/null # don't care about any output
\end{code}

The pipe operator has a fascinating history. The UNIX philosophy is writing small programs and chaining them together to do new and interesting things. Back in the early days, hard disk space was limited and write times were slow. Brian Kernighan wanted to maintain the philosophy while also not having to write a bunch of intermediate files that take up hard drive space. So, the UNIX pipe was born. A pipe take the \keyword{stdout} of the program on its left and feeds it to the \keyword{stdin} of the program on its write. Consider the command \keyword{tee}. It can be used as a replacement for the redirection operators because tee will both write to a file and output to standard out. It also has the added benefit that it doesn't need to be the last command in the list. Meaning, that you can write an intermediate result and continue your piping.

\begin{code}[language=console]
$ ./program | tee output.txt # Overwrite
$ ./program | tee -a output.txt # Append
$ head output.txt | wc | head -n 1 # Multi pipes
$ ((head output.txt) | wc) | head -n 1 # Same as above
$ ./program | tee intermediate.txt | wc
\end{code}

The \keyword{&&} and \keyword{||} operator 

\subsection{Tips and tricks}

\begin{comment}

\subsection{\texorpdfstring{What if I really really want \keyword{printf}
to call \keyword{write} without a
newline?}{What if I really really want printf to call write without a newline?}}\label{what-if-i-really-really-want-printf-to-call-write-without-a-newline}

Use \keyword{fflush(\ FILE*\ inp\ )}. The contents of the file will be
written. If I wanted to write ``Hello World'' with no newline, I could
write it like this.

\begin{code}[language=C]
int main(){
    fprintf(stdout, "Hello World");
    fflush(stdout);
    return 0;
}
\end{code}

\subsection{\texorpdfstring{How is \keyword{perror}
helpful?}{How is perror helpful?}}\label{how-is-perror-helpful}

Let's say that you have a function call that just failed (because you
checked the man page and it is a failing return code).
\keyword{perror(const\ char*\ message)} will print the English version of
the error to stderr

\begin{code}[language=C]
int main(){
    int ret = open("IDoNotExist.txt", O_RDONLY);
    if(ret < 0){
        perror("Opening IDoNotExist:");
    }
    //...
    return 0;
}
\end{code}

\section{Parsing Input}\label{parsing-input}

\subsection{How do I parse numbers from
strings?}\label{how-do-i-parse-numbers-from-strings}

Use
\keyword{long\ int\ strtol(const\ char\ *nptr,\ char\ **endptr,\ int\ base);}
or
\keyword{long\ long\ int\ strtoll(const\ char\ *nptr,\ char\ **endptr,\ int\ base);}.

What these functions do is take the pointer to your string
\keyword{*nptr} and a \keyword{base} (ie binary, octal, decimal,
hexadecimal etc) and an optional pointer \keyword{endptr} and returns a
parsed value.

\begin{code}[language=C]
int main(){
    const char *nptr = "1A2436";
    char* endptr;
    long int result = strtol(nptr, &endptr, 16);
    return 0;
}
\end{code}

Be careful though! Error handling is tricky because the function won't
return an error code. If you give it a string that is not a number it
will return 0. This means you cant differentiate between a valid ``0''
and an invalid string. See the man page for more details on strol
behavior with invalid and out of bounds values. A safer alternative is
use to \keyword{sscanf} (and check the return value).

\begin{code}[language=C]
int main(){
    const char *input = "0"; // or "!##@" or ""
    char* endptr;
    long int parsed = strtol(input, &endptr, 10);
    if(parsed == 0){
        // Either the input string was not a valid base-10 number or it really was zero!

    }
    return 0;
}
\end{code}

\subsection{\texorpdfstring{How do I parse input using \keyword{scanf}
into
parameters?}{How do I parse input using scanf into parameters?}}\label{how-do-i-parse-input-using-scanf-into-parameters}

Use \keyword{scanf} (or \keyword{fscanf} or \keyword{sscanf}) to get input
from the default input stream, an arbitrary file stream or a C string
respectively. It's a good idea to check the return value to see how many
items were parsed. \keyword{scanf} functions require valid pointers. It's
a common source of error to pass in an incorrect pointer value. For
example,

\begin{code}[language=C]
int *data = (int *) malloc(sizeof(int));
char *line = "v 10";
char type;
// Good practice: Check scanf parsed the line and read two values:
int ok = 2 == sscanf(line, "%c %d", &type, &data); // pointer error
\end{code}

We wanted to write the character value into c and the integer value into
the malloc'd memory. However, we passed the address of the data pointer,
not what the pointer is pointing to! So \keyword{sscanf} will change the
pointer itself. i.e.~the pointer will now point to address 10 so this
code will later fail e.g.~when free(data) is called.

\subsection{How do I stop scanf from causing a buffer
overflow?}\label{how-do-i-stop-scanf-from-causing-a-buffer-overflow}

The following code assumes the scanf won't read more than 10 characters
(including the terminating byte) into the buffer.

\begin{code}[language=C]
char buffer[10];
scanf("%s",buffer);
\end{code}

You can include an optional integer to specify how many characters
EXCLUDING the terminating byte:

\begin{code}[language=C]
char buffer[10];
scanf("%9s", buffer); // reads up to 9 charactes from input (leave room for the 10th byte to be the terminating byte)
\end{code}


\section{Memory mistakes}\label{memory-mistakes}

\subsection{Buffer overflow/ underflow}\label{buffer-overflow-underflow}

\begin{code}[language=C]
#define N (10)
int i = N, array[N];
for( ; i >= 0; i--) array[i] = i;
\end{code}

C does not check that pointers are valid. The above example writes into
\keyword{array{[}10{]}} which is outside the array bounds. This can cause
memory corruption because that memory location is probably being used
for something else. In practice, this can be harder to spot because the
overflow/underflow may occur in a library call e.g.

\begin{code}[language=C]
gets(array); // Let's hope the input is shorter than my array!
\end{code}


\paragraph{\texorpdfstring{Strings require \keyword{strlen(s)+1}
bytes}{Strings require strlen(s)+1 bytes}}\label{strings-require-strlens1-bytes}

Every string must have a null byte after the last characters. To store
the string ``Hi'' it takes 3 bytes: {[}H{]} {[}i{]} {[}\0{]}.

\begin{code}[language=C]
  char *strdup(const char *input) {  /* return a copy of 'input' */
    char *copy;
    copy = malloc(sizeof(char*));     /* nope! this allocates space for a pointer, not a string */
    copy = malloc(strlen(input));     /* Almost...but what about the null terminator? */
    copy = malloc(strlen(input) + 1); /* That's right. */
    strcpy(copy, input);   /* strcpy will provide the null terminator */
    return copy;
}
\end{code}

\subsection{Using uninitialized
variables}\label{using-uninitialized-variables}

\begin{code}[language=C]
int myfunction() {
  int x;
  int y = x + 2;
...
\end{code}

Automatic variables hold garbage (whatever bit pattern happened to be in
memory). It is an error to assume that it will always be initialized to
zero.

\subsection{Assuming Uninitialized memory will be zeroed}\label{assuming-uninitialized-memory-will-be-zeroed}

\begin{code}[language=C]
void myfunct() {
   char array[10];
   char *p = malloc(10);
\end{code}

Automatic (temporary variables) are not automatically initialized to zero. Heap allocations using malloc are not automatically initialized to zero.

\section{Logic and Program flow
mistakes}\label{logic-and-program-flow-mistakes}

\subsection{Equal vs.~equality}\label{equal-vs.equality}

\begin{code}[language=C]
int answer = 3; // Will print out the answer.
if (answer = 42) { printf("I've solved the answer! It's %d", answer);}
\end{code}

\subsection{Undeclared or incorrectly prototyped
functions}\label{undeclared-or-incorrectly-prototyped-functions}

\begin{code}[language=C]
time_t start = time();
\end{code}

The system function `time' actually takes a parameter (a pointer to some
memory that can receive the time\_t structure). The compiler did not
catch this error because the programmer did not provide a valid function
prototype by including \keyword{time.h}

\subsection{Extra Semicolons}\label{extra-semicolons}

\begin{code}[language=C]
for(int i = 0; i < 5; i++) ; printf("I'm printed once");
while(x < 10); x++ ; // X is never incremented
\end{code}

However, the following code is perfectly OK.

\begin{code}[language=C]
for(int i = 0; i < 5; i++){
    printf("%d\n", i);;;;;;;;;;;;;
}
\end{code}

It is OK to have this kind of code, because the C language uses
semicolons (;) to separate statements. If there is no statement in
between semicolons, then there is nothing to do and the compiler moves
on to the next statement

\section{Other Gotchas}\label{other-gotchas}

\subsection{\texorpdfstring{Does \keyword{sizeof} do
anything?}{Does sizeof do anything?}}\label{does-sizeof-do-anything}

\begin{code}[language=C]
int a = 0;
size_t size = sizeof(a++);
printf("size: %lu, a: %d", size, a);
\end{code}

What does the code print out?

\begin{code}[language=C]
size: 4, a: 0
\end{code}

Because sizeof is not actually evaluated at runtime. The compiler
assigns the type of all expressions and discards the extra results of
the expression.

\section{Strings, Structs, and
Gotcha's}\label{strings-structs-and-gotchas}

\section{So what's a string?}\label{so-whats-a-string}

\begin{figure}[htbp]
\centering
\includegraphics{https://i.imgur.com/CgsxyZb.png}
\caption{String}
\end{figure}

In C we have
\href{https://en.wikipedia.org/wiki/Null-terminated_string}{Null
Terminated} strings rather than
\href{https://en.wikipedia.org/wiki/String_(computer_science)\#Length-prefixed}{Length
Prefixed} for historical reasons. What that means for your average
everyday programming is that you need to remember the null character! A
string in C is defined as a bunch of bytes until you reach `\0' or the
Null Byte.

\subsection{Two places for strings}\label{two-places-for-strings}

Whenever you define a constant string (ie one in the form
\keyword{char*\ str\ =\ "constant"}) That string is stored in the
\emph{data} or \emph{code} segment that is \textbf{read-only} meaning
that any attempt to modify the string will cause a segfault.

If one, however, \keyword{malloc}'s space, one can change that string to
be whatever they want.

\subsection{Memory Mismanagement}\label{memory-mismanagement}

One common gotcha is when you write the following

\begin{code}[language=C]
char* hello_string = malloc(14);
                       ___ ___ ___ ___ ___ ___ ___ ___ ___ ___ ___ ___ ___ ___
// hello_string ----> | g | a | r | b | a | g | e | g | a | r | b | a | g | e |
                                                                              
hello_string = "Hello Bhuvan!";
// (constant string in the text segment)
// hello_string ----> [ "H" , "e" , "l" , "l" , "o" , " " , "B" , "h" , "u" , "v" , "a" , "n" , "!" , "\0" ]
                       ___ ___ ___ ___ ___ ___ ___ ___ ___ ___ ___ ___ ___ ___
// memory_leak -----> | g | a | r | b | a | g | e | g | a | r | b | a | g | e |
                                                                              
hello_string[9] = 't'; //segfault!!
\end{code}

What did we do? We allocated space for 14 bytes, reassigned the pointer
and successfully segfaulted! Remember to keep track of what your
pointers are doing. What you probably wanted to do was use a
\keyword{string.h} function \keyword{strcpy}.

\begin{code}[language=C]
strcpy(hello_string, "Hello Bhuvan!");
\end{code}

\subsection{Remember the NULL byte!}\label{remember-the-null-byte}

Forgetting to NULL terminate a string is a big affect on the strings!
Bounds checking is important. The heart bleed bug mentioned earlier in
the wiki book is partially because of this.

\subsection{Where can I find an In-Depth and Assignment-Comprehensive
explanation of all of these
functions?}\label{where-can-i-find-an-in-depth-and-assignment-comprehensive-explanation-of-all-of-these-functions}

\href{https://linux.die.net/man/3/string}{Right Here!}

\subsection{\texorpdfstring{String Information/Comparison:
\keyword{strlen}
\keyword{strcmp}}{String Information/Comparison: strlen strcmp}}\label{string-informationcomparison-strlen-strcmp}

\keyword{int\ strlen(const\ char\ *s)} returns the length of the string
not including the null byte

\keyword{int\ strcmp(const\ char\ *s1,\ const\ char\ *s2)} returns an
integer determining the lexicographic order of the strings. If s1 where
to come before s2 in a dictionary, then a -1 is returned. If the two
strings are equal, then 0. Else, 1.

With most of these functions, they expect the strings to be readable and
not NULL but there is undefined behavior when you pass them NULL.

\subsection{\texorpdfstring{String Alteration: \keyword{strcpy}
\keyword{strcat}
\keyword{strdup}}{String Alteration: strcpy strcat strdup}}\label{string-alteration-strcpy-strcat-strdup}

\keyword{char\ *strcpy(char\ *dest,\ const\ char\ *src)} Copies the
string at \keyword{src} to \keyword{dest}. \textbf{assumes dest has enough
space for src}

\keyword{char\ *strcat(char\ *dest,\ const\ char\ *src)} Concatenates the
string at \keyword{src} to the end of destination. \textbf{This function
assumes that there is enough space for \keyword{src} at the end of
destination including the NULL byte}

\keyword{char\ *strdup(const\ char\ *dest)} Returns a \keyword{malloc}'ed
copy of the string.

\subsection{\texorpdfstring{String Search: \keyword{strchr}
\keyword{strstr}}{String Search: strchr strstr}}\label{string-search-strchr-strstr}

\keyword{char\ *strchr(const\ char\ *haystack,\ int\ needle)} Returns a
pointer to the first occurrence of \keyword{needle} in the
\keyword{haystack}. If none found, \keyword{NULL} is returned.

\keyword{char\ *strstr(const\ char\ *haystack,\ const\ char\ *needle)}
Same as above but this time a string!

\subsection{\texorpdfstring{String Tokenize:
\keyword{strtok}}{String Tokenize: strtok}}\label{string-tokenize-strtok}

A dangerous but useful function strtok takes a string and tokenizes it.
Meaning that it will transform the strings into separate strings. This
function has a lot of specs so please read the man pages a contrived
example is below.

\begin{code}[language=C]
#include <stdio.h>
#include <string.h>

int main(){
    char* upped = strdup("strtok,is,tricky,!!");
    char* start = strtok(upped, ",");
    do{
        printf("%s\n", start);
    }while((start = strtok(NULL, ",")));
    return 0;
}
\end{code}

\textbf{Output}

\begin{code}[language=C]
strtok
is
tricky
!!
\end{code}

What happens when I change \keyword{upped} like this?

\begin{code}[language=C]
char* upped = strdup("strtok,is,tricky,,,!!");
\end{code}

\section{\texorpdfstring{So what's a \keyword{struct}?}{So what's a struct?}}\label{so-whats-a-struct}

In low-level terms, a struct is just a piece of contiguous memory,
nothing more. Just like an array, a struct has enough space to keep all
of its members. But unlike an array, it can store different types.
Consider the contact struct declared above

\begin{code}[language=C]
struct contact {
    char firstname[20];
    char lastname[20];
    unsigned int phone;
};

struct contact bhuvan;
\end{code}

\textbf{Brief aside}

\begin{code}[language=C]
/* a lot of times we will do the following typdef
 so we can just write contact contact1 */

typedef struct contact contact;
contact bhuvan;

/* You can also declare the struct like this to get
 it done in one statement */
typedef struct optional_name {
    ...
} contact;
\end{code}

If you compile the code without any optimizations and reordering, you
can expect the addresses of each of the variables to look like this.

\begin{code}[language=C]
&bhuvan           // 0x100
&bhuvan.firstname // 0x100 = 0x100+0x00
&bhuvan.lastname  // 0x114 = 0x100+0x14
&bhuvan.phone     // 0x128 = 0x100+0x28
\end{code}

Because all your compiler does is say `hey reserve this much space, and
I will go and calculate the offsets of whatever variables you want to
write to'. The offsets are where the variable starts at. The phone variables starts
at the \keyword{0x128}th bytes and continues for sizeof(int) bytes, but
not always. \textbf{Offsets don't determine where the variable ends
though}. Consider the following hack that you see in a lot of kernel
code.

\begin{code}[language=C]

typedef struct {
    int length;
    char c_str[0];
} string;

const char* to_convert = "bhuvan";
int length = strlen(to_convert);

// Let's convert to a c string
string* bhuvan_name;
bhuvan_name = malloc(sizeof(string) + length+1);
/*
Currently, our memory looks like this with junk in those black spaces
                ___ ___ ___ ___ ___ ___ ___ ___ ___ ___ ___
 bhuvan_name = |   |   |   |   |   |   |   |   |   |   |   |
                                                           
*/


bhuvan_name->length = length;
/*
This writes the following values to the first four bytes
The rest is still garbage
                ___ ___ ___ ___ ___ ___ ___ ___ ___ ___ ___
 bhuvan_name = | 0 | 0 | 0 | 6 |   |   |   |   |   |   |   |
                                                           
*/


strcpy(bhuvan_name->c_str, to_convert);
/*
Now our string is filled in correctly at the end of the struct

                ___ ___ ___ ___ ___ ___ ___ ___ ___ ___ ____
 bhuvan_name = | 0 | 0 | 0 | 6 | b | h | u | v | a | n | \0 |
                                                           ‾
*/

strcmp(bhuvan_name->c_str, "bhuvan") == 0 //The strings are equal!
\end{code}

\end{comment}

\section{Common Bugs}

\begin{code}[language=C]
void mystrcpy(char*dest, char* src) { 
  // void means no return value   
  while( *src ) { dest = src; src ++; dest++; }  
}
\end{code}

In the above code it simply changes the dest pointer to point to source
string. Also the nuls bytes are not copied. Here's a better version -

\begin{code}[language=C]
  while( *src ) { *dest = *src; src ++; dest++; } 
  *dest = *src;
\end{code}

Note it's also usual to see the following kind of implementation, which
does everything inside the expression test, including copying the nul
byte.

\begin{code}[language=C]
  while( (*dest++ = *src++ )) {};
\end{code}

\section{Topics}

\begin{itemize}
\tightlist
\item
  C Strings representation
\item
  C Strings as pointers
\item
  char p{[}{]}vs char* p
\item
  Simple C string functions (strcmp, strcat, strcpy)
\item
  sizeof char
\item
  sizeof x vs x*
\item
  Heap memory lifetime
\item
  Calls to heap allocation
\item
  Deferencing pointers
\item
  Address-of operator
\item
  Pointer arithmetic
\item
  String duplication
\item
  String truncation
\item
  double-free error
\item
  String literals
\item
  Print formatting.
\item
  memory out of bounds errors
\item
  static memory
\item
  fileio POSIX vs.~C library
\item
  C io fprintf and printf
\item
  POSIX file IO (read, write, open)
\item
  Buffering of stdout
\end{itemize}

\section{Questions/Exercises}

\begin{itemize}
\item
  What does the following print out?

\begin{code}[language=C]
int main(){
fprintf(stderr, "Hello ");
fprintf(stdout, "It's a small ");
fprintf(stderr, "World\n");
fprintf(stdout, "place\n");
return 0;
}
\end{code}
\item
  What are the differences between the following two declarations? What
  does \keyword{sizeof} return for one of them?

\begin{code}[language=C]
char str1[] = "bhuvan";
char *str2 = "another one";
\end{code}
\item
  What is a string in c?
\item
  Code up a simple \keyword{my\_strcmp}. How about \keyword{my\_strcat},
  \keyword{my\_strcpy}, or \keyword{my\_strdup}? Bonus: Code the functions
  while only going through the strings \emph{once}.
\item
  What should the following usually return?

\begin{code}[language=C]
int *ptr;
sizeof(ptr);
sizeof(*ptr);
\end{code}
\item
  What is \keyword{malloc}? How is it different than \keyword{calloc}.
  Once memory is \keyword{malloc}ed how can I use \keyword{realloc}?
\item
  What is the \keyword{\&} operator? How about \keyword{*}?
\item
  Pointer Arithmetic. Assume the following addresses. What are the
  following shifts?

\begin{code}[language=C]
char** ptr = malloc(10); //0x100
ptr[0] = malloc(20); //0x200
ptr[1] = malloc(20); //0x300
\end{code}

  \begin{itemize}
  \tightlist
  \item
    \keyword{ptr\ +\ 2}
  \item
    \keyword{ptr\ +\ 4}
  \item
    \keyword{ptr{[}0{]}\ +\ 4}
  \item
    \keyword{ptr{[}1{]}\ +\ 2000}
  \item
    \keyword{*((int)(ptr\ +\ 1))\ +\ 3}
  \end{itemize}
\item
  How do we prevent double free errors?
\item
  What is the printf specifier to print a string, \keyword{int}, or
  \keyword{char}?
\item
  Is the following code valid? If so, why? Where is \keyword{output}
  located?

\begin{code}[language=C]
char *foo(int var){
static char output[20];
snprintf(output, 20, "%d", var);
return output;
}
\end{code}
\item
  Write a function that accepts a string and opens that file prints out
  the file 40 bytes at a time but every other print reverses the string
  (try using POSIX API for this).
\item
  What are some differences between the POSIX filedescriptor model and
  C's \keyword{FILE*} (ie what function calls are used and which is
  buffered)? Does POSIX use C's \keyword{FILE*} internally or vice versa?
\end{itemize}

\bibliographystyle{plainnat}
\bibliography{introc/introc}
