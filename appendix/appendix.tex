\section{The Hitchhiker's Guide to Debugging C Programs}

This is going to be a massive guide to helping you debug your C programs. There are different levels that you can check errors and we will be going through most of them. Feel free to add anything that you found helpful in debugging C programs including but not limited to, debugger usage, recognizing common error types, gotchas, and effective googling tips.

\subsection{In-Code Debugging}

\subsubsection{Clean code}

Make your code modular using helper functions. If there is a repeated task (getting the pointers to contiguous blocks in the malloc MP, for example), make them helper functions. And make sure each function does one thing very well, so that you don't have to debug twice.

Let's say that we are doing selection sort by finding the minimum element each iteration like so,

\begin{lstlisting}[language=C]
void selection_sort(int *a, long len){
     for(long i = len-1; i > 0; --i){
         long max_index = i;
         for(long j = len-1; j >= 0; --j){
             if(a[max_index] < a[j]){
                  max_index = j;
             }
         }
         int temp = a[i];
         a[i] = a[max_index];
         a[max_index] = temp;
     }

}
\end{lstlisting}

Many can see the bug in the code, but it can help to refactor the above
method into

\begin{lstlisting}[language=C]
long max_index(int *a, long start, long end);
void swap(int *a, long idx1, long idx2);
void selection_sort(int *a, long len);
\end{lstlisting}

And the error is specifically in one function. In the end, we are not a class about refactoring/debugging your code. In fact, most kernel code is so atrocious that you don't want to read it. But for the sake of debugging, it may benefit you in the long run to adopt some of these practices.

\subsubsection{Asserts!}\label{asserts}

Use assertions to make sure your code works up to a certain point -- and importantly, to make sure you don't break it later. For example, if your data structure is a doubly linked list, you can do something like \texttt{assert(node-\textgreater{}size\ ==\ node-\textgreater{}next-\textgreater{}prev-\textgreater{}size)} to assert that the next node has a pointer to the current node. You can also check the pointer is pointing to an expected range of memory address, not null, -\textgreater{}size is reasonable etc. The \texttt{NDEBUG} macro will disable all assertions, so don't forget to set that once you finish debugging. \href{http://www.cplusplus.com/reference/cassert/assert/}{assert link}

Here's a quick example with assert. Let's say that I'm writing code using memcpy

\begin{lstlisting}[language=C]
assert(!(src < dest+n && dest < src+n)); //Checks overlap
memcpy(dest, src, n);
\end{lstlisting}

This check can be turned off at compile time, but will save you \textbf{tons} of trouble debugging!

\subsubsection{printfs}\label{printfs}

When all else fails, print like crazy! Each of your functions should have an idea of what it is going to do (ie find\_min better find the minimum element). You want to test that each of your functions is doing what it set out to do and see exactly where your code breaks. In the case with race conditions, tsan may be able to help, but having each thread print out data at certain times could help you identify the race condition.

\section{Valgrind}\label{valgrind}

Valgrind is a suite of tools designed to provide debugging and profiling tools to make your programs more correct and detect some runtime issues. The most used of these tools is Memcheck, which can detect many memory-related errors that are common in C and C++ programs and that can lead to crashes and unpredictable behaviour (for example, unfreed memory buffers). To run Valgrind on your program:

\begin{lstlisting}[language=C]
valgrind --leak-check=full --show-leak-kinds=all myprogram arg1 arg2
\end{lstlisting}

Arguments are optional and the default tool that will run is Memcheck. The output will be presented in form of number of allocations, number of freed allocations, and the number of errors.

Suppose we have a simple program like this:

\begin{lstlisting}[language=C]
#include <stdlib.h>

void dummy_function() {
	int* x = malloc(10 * sizeof(int));
	x[10] = 0;        // error 1:as you can see here we write to an out of bound memory address
}                    // error 2: memory leak the allocated x not freed

int main(void) {
	dummy_function();
	return 0;
}
\end{lstlisting}

This program compiles and run with no errors. Let's see what Valgrind will output.

\begin{lstlisting}[language=C]
==29515== Memcheck, a memory error detector
==29515== Copyright (C) 2002-2015, and GNU GPL'd, by Julian Seward et al.
==29515== Using Valgrind-3.11.0 and LibVEX; rerun with -h for copyright info
==29515== Command: ./a
==29515== 
==29515== Invalid write of size 4
==29515==    at 0x400544: dummy_function (in /home/rafi/projects/exocpp/a)
==29515==    by 0x40055A: main (in /home/rafi/projects/exocpp/a)
==29515==  Address 0x5203068 is 0 bytes after a block of size 40 alloc'd
==29515==    at 0x4C2DB8F: malloc (in /usr/lib/valgrind/vgpreload_memcheck-amd64-linux.so)
==29515==    by 0x400537: dummy_function (in /home/rafi/projects/exocpp/a)
==29515==    by 0x40055A: main (in /home/rafi/projects/exocpp/a)
==29515== 
==29515== 
==29515== HEAP SUMMARY:
==29515==     in use at exit: 40 bytes in 1 blocks
==29515==   total heap usage: 1 allocs, 0 frees, 40 bytes allocated
==29515== 
==29515== LEAK SUMMARY:
==29515==    definitely lost: 40 bytes in 1 blocks
==29515==    indirectly lost: 0 bytes in 0 blocks
==29515==      possibly lost: 0 bytes in 0 blocks
==29515==    still reachable: 0 bytes in 0 blocks
==29515==         suppressed: 0 bytes in 0 blocks
==29515== Rerun with --leak-check=full to see details of leaked memory
==29515== 
==29515== For counts of detected and suppressed errors, rerun with: -v
==29515== ERROR SUMMARY: 1 errors from 1 contexts (suppressed: 0 from 0)
\end{lstlisting}

\textbf{Invalid write}: It detected our heap block overrun, writing outside of allocated block.

\textbf{Definitely lost}: Memory leak --- you probably forgot to free a memory block.

Valgrind is a very effective tool to check for errors at runtime. C is very special when it comes to such behavior, so after compiling your program you can use Valgrind to fix errors that your compiler may not catch and that usually happen when your program is running.

For more information, you can refer to the \href{http://valgrind.org/docs/manual/quick-start.html}{valgrind website}.

\subsection{TSAN}

ThreadSanitizer is a tool from Google, built into clang and gcc, to help you detect race conditions in your code. For more information about the tool, see the Github wiki. Note, that running with tsan will slow your code down a bit. Consider the following code.

\begin{lstlisting}[language=C]
#include <pthread.h>
#include <stdio.h>

int global;

void *Thread1(void *x) {
    global++;
    return NULL;
}

int main() {
    pthread_t t[2];
    pthread_create(&t[0], NULL, Thread1, NULL);
    global = 100;
    pthread_join(t[0], NULL);
}
// compile with gcc -fsanitize=thread -pie -fPIC -ltsan -g simple_race.c
\end{lstlisting}

We can see that there is a race condition on the variable \texttt{global}. Both the main thread and the thread created with \textt{pthread\_create} will try to change the value at the same time. But, does ThreadSantizer catch it?

\begin{lstlisting}[language=C]
$ ./a.out
==================
WARNING: ThreadSanitizer: data race (pid=28888)
  Read of size 4 at 0x7f73ed91c078 by thread T1:
    #0 Thread1 /home/zmick2/simple_race.c:7 (exe+0x000000000a50)
    #1  :0 (libtsan.so.0+0x00000001b459)

  Previous write of size 4 at 0x7f73ed91c078 by main thread:
    #0 main /home/zmick2/simple_race.c:14 (exe+0x000000000ac8)

  Thread T1 (tid=28889, running) created by main thread at:
    #0  :0 (libtsan.so.0+0x00000001f6ab)
    #1 main /home/zmick2/simple_race.c:13 (exe+0x000000000ab8)

SUMMARY: ThreadSanitizer: data race /home/zmick2/simple_race.c:7 Thread1
==================
ThreadSanitizer: reported 1 warnings
\end{lstlisting}

If we compiled with the debug flag, then it would give us the variable name as well.

\section{GDB}\label{gdb}

\href{http://www.cs.cmu.edu/~gilpin/tutorial/}{Introduction to gdb}

\paragraph{Setting breakpoints programmatically}

A very useful trick when debugging complex C programs with GDB is setting breakpoints in the source code.

\begin{lstlisting}[language=C]
int main() {
    int val = 1;
    val = 42;
    asm("int $3"); // set a breakpoint here
    val = 7;
}
\end{lstlisting}

\begin{lstlisting}[language=C]
$ gcc main.c -g -o main && ./main
(gdb) r
[...]
Program received signal SIGTRAP, Trace/breakpoint trap.
main () at main.c:6
6     val = 7;
(gdb) p val
$1 = 42
\end{lstlisting}

\paragraph{Checking memory content}\label{checking-memory-content}

\href{http://www.delorie.com/gnu/docs/gdb/gdb\_56.html}{Memory Content}

For example,

\begin{lstlisting}[language=C]
int main() {
    char bad_string[3] = {'C', 'a', 't'};
    printf("%s", bad_string);
}
\end{lstlisting}

\begin{lstlisting}
$ gcc main.c -g -o main && ./main
$ Cat ZVQ� $
\end{lstlisting}

\begin{lstlisting}
(gdb) l
1 #include <stdio.h>
2 int main() {
3     char bad_string[3] = {'C', 'a', 't'};
4     printf("%s", bad_string);
5 }
(gdb) b 4
Breakpoint 1 at 0x100000f57: file main.c, line 4.
(gdb) r
[...]
Breakpoint 1, main () at main.c:4
4     printf("%s", bad_string);
(gdb) x/16xb bad_string
0x7fff5fbff9cd: 0x63  0x61  0x74  0xe0  0xf9  0xbf  0x5f  0xff
0x7fff5fbff9d5: 0x7f  0x00  0x00  0xfd  0xb5  0x23  0x89  0xff

(gdb)
\end{lstlisting}

Here, by using the \texttt{x} command with parameters \texttt{16xb}, we can see that starting at memory address \texttt{0x7fff5fbff9c} (value of \texttt{bad\_string}), \texttt{printf} would actually see the following sequence of bytes as a string because we provided a malformed string without a null terminator.

\section{Life in the terminal}

\texttt{0x43\ 0x61\ 0x74\ 0xe0\ 0xf9\ 0xbf\ 0x5f\ 0xff\ 0x7f\ 0x00}

\section{System Programming Jokes}

Warning: Authors are not responsible for any neuro-apoptosis caused by these ``jokes.'' - Groaners are allowed.

\subsection{Light bulb jokes}

Q. How many system programmers does it take to change a lightbulb?

A. Just one but they keep changing it until it returns zero.

A. None they prefer an empty socket.

A. Well you start with one but actually it waits for a child to do all of the work.

\subsection{Groaners}

Why did the baby system programmer like their new colorful blankie? It was multithreaded.

Why are your programs so fine and soft? I only use 400-thread-count or higher programs.

Where do bad student shell processes go when they die? Forking Hell.

Why are C programmers so messy? They store everything in one big heap.

\subsection{System Programmer (Definition)}

A system programmer is\ldots{}

Someone who knows \texttt{sleepsort} is a bad idea but still dreams of an excuse to use it.

Someone who never lets their code deadlock\ldots{} but when it does, causes more problems than everyone else combined.

Someone who believes zombies are real.

Someone who doesn't trust their process to run correctly without testing with the same data, kernel, compiler, RAM, filesystem size,file system format, disk brand, core count, CPU load, weather, magnetic flux, orientation, pixie dust, horoscope sign, wall color, wall gloss and reflectance, motherboard, vibration, illumination, backup battery, time of day, temperature, humidity, lunar position, sun-moon, co-position\ldots{}

\subsection{System Program (Definition)}\label{system-program-definition}

A system program \ldots{}

Evolves until it can send email.

Evolves until it has the potential to create, connect and kill other programs and consume all possible CPU,memory,network,\ldots{} resources on all possible devices but chooses not to. Today.
