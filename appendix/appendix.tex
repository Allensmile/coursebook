\chapter{Appendix}

\section{Shell}

A shell is actually how you are going to be interacting with the system. Before user friendly operating systems, when a computer started up all you had access to was a shell. This meant that all of your commands and editing had to be done this way. Nowadays, our computers start up in desktop mode, but one can still access a shell using a terminal.

\begin{lstlisting}[language=bash]
(Stuff) $
\end{lstlisting}

It is ready for your next command! You can type in a lot of Unix utilities like \keyword{ls}, \keyword{echo\ Hello} and the shell will execute them and give you the result. Some of these are what are known as \keyword{shell-builtins} meaning that the code is in the shell program itself. Some of these are compiled programs that you run. The shell only looks through a special variable called path which contains a list of \keyword{:} separated paths to search for an executable with your name, here is an example path.

\begin{lstlisting}[language=bash]
$ echo $PATH
/usr/local/sbin:/usr/local/bin:/usr/sbin:
/usr/bin:/sbin:/bin:/usr/games:/usr/local/games
\end{lstlisting}

So when the shell executes \keyword{ls}, it looks through all of those directories, finds \keyword{/bin/ls} and executes that.

\begin{lstlisting}[language=bash]
$ ls
...
$ /bin/ls
\end{lstlisting}

You can always call through the full path. That is always why in past classes if you want to run something on the terminal you've had to do \keyword{./exe} because typically the directory that you are working in is not in the \keyword{PATH} variable. The \keyword{.} expands to your current directory and your shell executes \keyword{\textless{}current\_dir\textgreater{}/exe} which is a valid command.

\subsection{Shell tricks and tips}

\begin{itemize}
\item The up arrow will get you your most recent command
\item \keyword{ctrl-r} will search commands that you previously ran
\item \keyword{ctrl-c} will interrupt your shell's process
\item \keyword{!!} will execute the last command
\item \keyword{!<num>} goes back that many commands and runs that
\item \keyword{!<prefix>} runs the last command that has that prefix
\item \keyword{!\$} is the last arg of the previous command
\item \keyword{!*} is all args of the previous command
\item \keyword{\^pat\^sub} takes the last command and substitutes the pattern pat for the substitution sub
\item \keyword{cd -} goes to the previous directory
\item \keyword{pushd <dir>} pushes the current directory on a stack and cds
\item \keyword{popd} cds to the directory at the top of the stack
\end{itemize}

\subsection{Alright then what's a terminal?}

A terminal is just an application that displays the output form the shell. You can have your default terminal, a quake based terminal, terminator, the options are endless!

\subsection{Common Utilities}

\begin{enumerate}
\item \keyword{cat} concatenate multiple files. It is regularly used to print out the contents of a file to the terminal but the original use was concatenation.

\begin{lstlisting}[language=bash]
$ cat file.txt
...
$ cat shakespeare.txt shakespeare.txt > two_shakes.txt
\end{lstlisting}

\item \keyword{diff} tells you the difference of the two files. If nothing is printed, then zero is returned meaning the files are the same byte for byte. Otherwise, the longest common subsequence difference is printed

\begin{lstlisting}[language=bash]
$ cat prog.txt
hello
world
$ cat adele.txt
hello
it's me
$ diff prog.txt prog.txt
$ diff shakespeare.txt shakespeare.txt
2c2
< world
---
> it's me
\end{lstlisting}

\item \keyword{grep} tells you which lines in a file or standard in match a POSIX pattern.

\begin{lstlisting}[language=bash]
$ grep it adele.txt
it's me
\end{lstlisting}

\item \keyword{ls} tells you which files are in the current directory.
\item \keyword{cd} this is a shell builtin but it changes to a relative or absolute directory

\begin{lstlisting}[language=bash]
$ cd /usr
$ cd lib/
$ cd -
$ pwd
/usr/
\end{lstlisting}

\item \keyword{man} every system programmers favorite command, tells you more about all your favorite functions!
\item \keyword{make} executes programs according to a makefile.

\end{enumerate}

\subsection{Syntactic}

Shells have many useful utilities like saving output to a file using redirection \keyword{>}. This overwrites the file from the beginning. If you only meant to append to the file, you can use \keyword{>>}. Unix also allows file descriptor swapping. This means that you can take the output going to one file descriptor and make it seem like its coming out of another. The most common one is \keyword{2>&1} which means take the stderr and make it seem like it is coming out of standard out. This is important because when you use \keyword{>} and \keyword{>>} they only write the standard output of the file. There are some examples below.

\begin{lstlisting}[language=bash]
$ ./program > output.txt # To overwrite
$ ./program >> output.txt # To append
$ ./program 2>&1 > output_all.txt # stderr & stdout
$ ./program 2>&1 > /dev/null # don't care about any output
\end{lstlisting}

The pipe operator has a fascinating history. The UNIX philosophy is writing small programs and chaining them together to do new and interesting things. Back in the early days, hard disk space was limited and write times were slow. Brian Kernighan wanted to maintain the philosophy while also not having to write a bunch of intermediate files that take up hard drive space. So, the UNIX pipe was born. A pipe take the \keyword{stdout} of the program on its left and feeds it to the \keyword{stdin} of the program on its write. Consider the command \keyword{tee}. It can be used as a replacement for the redirection operators because tee will both write to a file and output to standard out. It also has the added benefit that it doesn't need to be the last command in the list. Meaning, that you can write an intermediate result and continue your piping.

\begin{lstlisting}[language=bash]
$ ./program | tee output.txt # Overwrite
$ ./program | tee -a output.txt # Append
$ head output.txt | wc | head -n 1 # Multi pipes
$ ((head output.txt) | wc) | head -n 1 # Same as above
$ ./program | tee intermediate.txt | wc
\end{lstlisting}

The \keyword{&&} and \keyword{||} operator are operators that execute a command sequentially. \keyword{&&} only executes a command if the previous command succeeds, and \keyword{||} always executes the next command.

\begin{lstlisting}[language=bash]
$ false && echo "Hello!"
$ true && echo "Hello!"
$ false || echo "Hello!"
\end{lstlisting}

\subsection{What are environment variables?}

Well each process gets its own dictionary of environment variables that are copied over to the child. Meaning, if the parent changes their environment variables it won't be transferred to the child and vice versa. This is important in the fork-exec-wait trilogy if you want to exec a program with different environment variables than your parent (or any other process).

For example, you can write a C program that loops through all of the time zones and executes the \keyword{date} command to print out the date and time in all locals. Environment variables are used for all sorts of programs so modifying them is important.

\section{Stack Smashing}

Each thread uses a stack memory. The stack `grows downwards' - if a function calls another function, then the stack is extended to smaller memory addresses. Stack memory includes non-static automatic (temporary) variables, parameter values and the return address. If a buffer is too small some data (e.g.~input values from the user), then there is a real possibility that other stack variables and even the return address will be overwritten. The precise layout of the stack's contents and order of the automatic variables is architecture and compiler dependent. However with a little investigative work we can learn how to deliberately smash the stack for a particular architecture.

The example below demonstrates how the return address is stored on the stack. For a particular 32 bit architecture \href{http://cs-education.github.io/sys/}{Live Linux Machine}, we determine that the return address is stored at an address two pointers (8 bytes) above the address of the automatic variable. The code deliberately changes the stack value so that when the input function returns, rather than continuing on inside the main method, it jumps to the exploit function instead.

\begin{lstlisting}[language=C]
// Overwrites the return address on the following machine:
// http://cs-education.github.io/sys/
#include <stdio.h>
#include <stdlib.h>
#include <unistd.h>

void breakout() {
  puts("Welcome. Have a shell...");
  system("/bin/sh");
}
void input() {
  void *p;
  printf("Address of stack variable: %p\n", &p);
  printf("Something that looks like a return address on stack: %p\n", *((&p)+2));
  // Let's change it to point to the start of our sneaky function.
  *((&p)+2) = breakout;
}
int main() {
  printf("main() code starts at %p\n",main);

  input();
  while (1) {
    puts("Hello");
    sleep(1);
  }

  return 0;
}
\end{lstlisting}

There are \href{https://en.wikipedia.org/wiki/Stack_buffer_overflow}{a lot} of ways that computers tend to get around this.

\section{Assorted Man Pages}

\subsection{Malloc}\label{man_malloc}

{
\begin{center}
\fontsize{10pt}{10pt}\selectfont

\begin{verbatim}
Copyright (c) 1993 by Thomas Koenig (ig25@rz.uni-karlsruhe.de)
%%%LICENSE_START(VERBATIM)
Permission is granted to make and distribute verbatim copies of this
manual provided the copyright notice and this permission notice are
preserved on all copies.

Permission is granted to copy and distribute modified versions of this
manual under the conditions for verbatim copying, provided that the
entire resulting derived work is distributed under the terms of a.
permission notice identical to this one.

Since the Linux kernel and libraries are constantly changing, this
manual page may be incorrect or out-of-date.  The author(s) assume no
responsibility for errors or omissions, or for damages resulting from
the use of the information contained herein.  The author(s) may not
have taken the same level of care in the production of this manual,
which is licensed free of charge, as they might when working
professionally.

Formatted or processed versions of this manual, if unaccompanied by
the source, must acknowledge the copyright and authors of this work.
%%%LICENSE_END

MALLOC(3)            Linux Programmer's Manual                MALLOC(3) 

NAME
       malloc, free, calloc, realloc - allocate and free dynamic memory

SYNOPSIS
       #include <stdlib.h>

       void *malloc(size_t size);
       void free(void *ptr);
       void *calloc(size_t nmemb, size_t size);
       void *realloc(void *ptr, size_t size);
       void *reallocarray(void *ptr, size_t nmemb, size_t size);

   Feature Test Macro Requirements for glibc (see feature_test_macros(7)):

       reallocarray():
           _GNU_SOURCE

DESCRIPTION
       The malloc() function allocates size bytes and returns a
       pointer to the allocated memory. The memory is not initialized.
       If size is 0, then malloc() returns either NULL, or     
       a unique pointer value that can later be successfully passed
       to free().

       The free() function frees the memory space pointed to by ptr,
       which must have been returned by a previous call to malloc(),
       calloc(), or realloc().  Otherwise, or if free(ptr)     
       has already been called before, undefined behavior occurs.
       If ptr is NULL, no operation is performed.

       The calloc() function allocates memory for an array of nmemb
       elements of size bytes each and returns a pointer to the
       allocated memory. The memory is set to zero. If nmemb or size
       is 0, then calloc() returns either NULL, or a unique pointer
       value that can later be successfully passed to free().

       The realloc() function changes the size of the memory block
       pointed to by ptr to size bytes. The contents will be unchanged
       in the range from the start of the region up to the minimum of
       the old and new sizes. If the new size is larger than the old
       size, the added memory will not be initialized. If ptr is NULL,
       then the call is equivalent to malloc(size), for all values of
       size; if size is equal to zero, and ptr is not NULL, then the
       call is equivalent to free(ptr). Unless ptr is NULL, it must
       have been returned by an earlier call to malloc(), calloc(), or
       realloc(). If the area pointed to was moved, a free(ptr) is done.

       The reallocarray() function changes the size of the memory block
       pointed to by ptr to be large enough for an array of nmemb
       elements, each of which is size bytes. It is equivalent to
       the call

               realloc(ptr, nmemb * size);

       However, unlike that realloc() call, reallocarray() fails
       safely in the case where the multiplication would overflow.
       If such an overflow occurs, reallocarray() returns NULL,
       sets errno to ENOMEM, and leaves the original block of memory
       unchanged.

RETURN VALUE
       The  malloc()  and  calloc() functions return a pointer to the
       allocated memory, which is suitably aligned for any built-in
       type. On error, these functions return NULL. NULL may also be
       returned by a successful call to malloc() with a size of zero,
       or by a successful call to calloc() with nmemb or size equal
       to zero.

       The free() function returns no value.

       The realloc() function returns a pointer to the newly allocated
       memory, which is suitably aligned for any built-in type and may
       be different from ptr, or NULL if the request fails. If size
       was equal to 0, either NULL or a pointer suitable to be passed
       to free() is returned. If realloc() fails, the original block is
       left untouched; it is not freed or moved.

       On success, the reallocarray() function returns a pointer to the
       newly allocated memory. On failure, it returns NULL and the
       original block of memory is left untouched.

ERRORS
       calloc(), malloc(), realloc(), and reallocarray() can fail with
       the following error:

       ENOMEM Out of memory. Possibly, the application hit the
       RLIMIT_AS or RLIMIT_DATA limit described in getrlimit(2).

ATTRIBUTES
       For an explanation of the terms used in this section, see
       attributes(7).

       +---------------------+---------------+---------+
       |Interface            | Attribute     | Value   |
       |-----------------------------------------------|
       |malloc(), free(),    | Thread safety | MT-Safe |
       |calloc(), realloc()  |               |         |
       +---------------------+---------------+---------+

CONFORMING TO
       malloc(), free(), calloc(), realloc(): POSIX.1-2001,
       POSIX.1-2008, C89, C99.

       reallocarray() is a nonstandard extension that first appeared in
       OpenBSD 5.6 and FreeBSD 11.0.

NOTES
       By default, Linux follows an optimistic memory allocation
       strategy. This means that when malloc() returns non-NULL there
       is no guarantee that the memory really is available. In case it
       turns out that the system is out of memory, one or more
       processes will be killed by the OOM killer. For more
       information, see the description of /proc/sys/vm/over-    
       commit_memory and /proc/sys/vm/oom_adj in proc(5), and the
       Linux kernel source file Documentation/vm/overcommit-accounting.

       Normally, malloc() allocates memory from the heap, and adjusts
       the size of the heap as required, using sbrk(2). When
       allocating blocks of memory larger than MMAP_THRESHOLD bytes,
       the glibc malloc() implementation allocates the memory as a
       private anonymous mapping using mmap(2).  MMAP_THRESHOLD is 128
       kB by default, but is adjustable using mallopt(3). Prior to
       Linux 4.7 allocations performed using mmap(2) were unaffected
       by the RLIMIT_DATA resource limit; since Linux 4.7, this limit
       is also enforced for allocations performed using mmap(2). 

       To avoid corruption in multithreaded applications, mutexes are
       used internally to protect the memory-management data structures
       employed by these functions. In a multithreaded application in
       which threads simultaneously allocate and free memory, there
       could be contention for these mutexes. To scalably handle
       memory allocation in multithreaded applications, glibc creates
       additional memory allocation arenas if mutex contention is
       detected. Each arena is a large region of memory that is
       internally allocated by the system (using brk(2) or mmap(2)),
       and managed with its own mutexes. 

       SUSv2 requires malloc(), calloc(), and realloc() to set errno to
       ENOMEM upon failure. Glibc assumes that this is done (and the
       glibc versions of these routines do  this); if you use a private
       malloc implementation that does not set errno, then certain
       library routines may fail without having a reason in errno.

       Crashes in malloc(), calloc(), realloc(), or free() are almost
       always related to heap corruption, such as overflowing an
       allocated chunk or freeing the same pointer twice. 

       The malloc() implementation is tunable via environment
       variables; see mallopt(3) for details. 

SEE ALSO
       valgrind(1), brk(2), mmap(2), alloca(3), malloc_get_state(3),
       malloc_info(3), malloc_trim(3), malloc_usable_size(3),
       mallopt(3), mcheck(3), mtrace(3), posix_memalign(3)
\end{verbatim}
\end{center}
}

\section{System Programming Jokes}

\keyword{0x43\ 0x61\ 0x74\ 0xe0\ 0xf9\ 0xbf\ 0x5f\ 0xff\ 0x7f\ 0x00}

Warning: Authors are not responsible for any neuro-apoptosis caused by these ``jokes.'' - Groaners are allowed.

\subsection{Light bulb jokes}

Q. How many system programmers does it take to change a lightbulb?

A. Just one but they keep changing it until it returns zero.

A. None they prefer an empty socket.

A. Well you start with one but actually it waits for a child to do all of the work.

\subsection{Groaners}

Why did the baby system programmer like their new colorful blankie? It was multithreaded.

Why are your programs so fine and soft? I only use 400-thread-count or higher programs.

Where do bad student shell processes go when they die? Forking Hell.

Why are C programmers so messy? They store everything in one big heap.

\subsection{System Programmer (Definition)}

A system programmer is\ldots{}

Someone who knows \keyword{sleepsort} is a bad idea but still dreams of an excuse to use it.

Someone who never lets their code deadlock\ldots{} but when it does, causes more problems than everyone else combined.

Someone who believes zombies are real.

Someone who doesn't trust their process to run correctly without testing with the same data, kernel, compiler, RAM, filesystem size,file system format, disk brand, core count, CPU load, weather, magnetic flux, orientation, pixie dust, horoscope sign, wall color, wall gloss and reflectance, motherboard, vibration, illumination, backup battery, time of day, temperature, humidity, lunar position, sun-moon, co-position\ldots{}

A system program \ldots{}

Evolves until it can send email.

Evolves until it has the potential to create, connect and kill other programs and consume all possible CPU,memory,network,\ldots{} resources on all possible devices but chooses not to. Today.
