The final exam will likely include multiple choice questions that test
your mastery of the following.

\begin{verbatim}
CSP (critical section problems)
HTTP
SIGINT
TCP
TLB
Virtual Memory
arrays
barrier
c strings
chmod
client/server
coffman conditions
condition variables
context switch
deadlock
dining philosophers
epoll
exit
file I/O
file system representation
fork/exec/wait
fprintf
free
heap allocator
heap/stack
inode vs name
malloc
mkfifo
mmap
mutexes
network ports
open/close
operating system terms
page fault
page tables
pipes
pointer arithmetic
pointers
printing (printf)
producer/consumer
progress/mutex
race conditions
read/write
reader/writer
resource allocation graphs
ring buffer
scanf 
buffering
scheduling
select
semaphores
signals
sizeof
stat
stderr/stdout
symlinks
thread control (_create, _join, _exit)
variable initializers
variable scope
vm thrashing
wait macros
write/read with errno, EINTR and partial data
\end{verbatim}

\section{C}\label{c}

\section{Memory and Strings}\label{memory-and-strings}

\subsection{Q1.1}\label{q1.1}

In the example below, which variables are guaranteed to print the value
of zero?

\begin{Shaded}
\begin{Highlighting}[]
\DataTypeTok{int} \NormalTok{a;}
\DataTypeTok{static} \DataTypeTok{int} \NormalTok{b;}

\DataTypeTok{void} \NormalTok{func() \{}
   \DataTypeTok{static} \DataTypeTok{int} \NormalTok{c;}
   \DataTypeTok{int} \NormalTok{d;}
   \NormalTok{printf(}\StringTok{"%d %d %d %d}\CharTok{\textbackslash{}n}\StringTok{"}\NormalTok{,a,b,c,d);}
\NormalTok{\}}
\end{Highlighting}
\end{Shaded}

\subsection{Q 1.2}\label{q-1.2}

In the example below, which variables are guaranteed to print the value
of zero?

\begin{Shaded}
\begin{Highlighting}[]
\DataTypeTok{void} \NormalTok{func() \{}
   \DataTypeTok{int}\NormalTok{* ptr1 = malloc( }\KeywordTok{sizeof}\NormalTok{(}\DataTypeTok{int}\NormalTok{) );}
   \DataTypeTok{int}\NormalTok{* ptr2 = realloc(NULL, }\KeywordTok{sizeof}\NormalTok{(}\DataTypeTok{int}\NormalTok{) );}
   \DataTypeTok{int}\NormalTok{* ptr3 = calloc( }\DecValTok{1}\NormalTok{, }\KeywordTok{sizeof}\NormalTok{(}\DataTypeTok{int}\NormalTok{) );}
   \DataTypeTok{int}\NormalTok{* ptr4 = calloc( }\KeywordTok{sizeof}\NormalTok{(}\DataTypeTok{int}\NormalTok{) , }\DecValTok{1}\NormalTok{);}
   
   \NormalTok{printf(}\StringTok{"%d %d %d %d}\CharTok{\textbackslash{}n}\StringTok{"}\NormalTok{,*ptr1,*ptr2,*ptr3,*ptr4);}
\NormalTok{\}}
\end{Highlighting}
\end{Shaded}

\subsection{Q 1.3}\label{q-1.3}

Explain the error in the following attempt to copy a string.

\begin{Shaded}
\begin{Highlighting}[]
\DataTypeTok{char}\NormalTok{* copy(}\DataTypeTok{char}\NormalTok{*src) \{}
 \DataTypeTok{char}\NormalTok{*result = malloc( strlen(src) ); }
 \NormalTok{strcpy(result, src); }
 \KeywordTok{return} \NormalTok{result;}
\NormalTok{\}}
\end{Highlighting}
\end{Shaded}

\subsection{Q 1.4}\label{q-1.4}

Why does the following attempt to copy a string sometimes work and
sometimes fail?

\begin{Shaded}
\begin{Highlighting}[]
\DataTypeTok{char}\NormalTok{* copy(}\DataTypeTok{char}\NormalTok{*src) \{}
 \DataTypeTok{char}\NormalTok{*result = malloc( strlen(src) +}\DecValTok{1} \NormalTok{); }
 \NormalTok{strcat(result, src); }
 \KeywordTok{return} \NormalTok{result;}
\NormalTok{\}}
\end{Highlighting}
\end{Shaded}

\subsection{Q 1.4}\label{q-1.4-1}

Explain the two errors in the following code that attempts to copy a
string.

\begin{Shaded}
\begin{Highlighting}[]
\DataTypeTok{char}\NormalTok{* copy(}\DataTypeTok{char}\NormalTok{*src) \{}
 \DataTypeTok{char} \NormalTok{result[}\KeywordTok{sizeof}\NormalTok{(src)]; }
 \NormalTok{strcpy(result, src); }
 \KeywordTok{return} \NormalTok{result;}
\NormalTok{\}}
\end{Highlighting}
\end{Shaded}

\subsection{Q 1.5}\label{q-1.5}

Which of the following is legal?

\begin{Shaded}
\begin{Highlighting}[]
\DataTypeTok{char} \NormalTok{a[] = }\StringTok{"Hello"}\NormalTok{; strcpy(a, }\StringTok{"World"}\NormalTok{);}
\DataTypeTok{char} \NormalTok{b[] = }\StringTok{"Hello"}\NormalTok{; strcpy(b, }\StringTok{"World12345"}\NormalTok{, b);}
\DataTypeTok{char}\NormalTok{* c = }\StringTok{"Hello"}\NormalTok{; strcpy(c, }\StringTok{"World"}\NormalTok{);}
\end{Highlighting}
\end{Shaded}

\subsection{Q 1.6}\label{q-1.6}

Complete the function pointer typedef to declare a pointer to a function
that takes a void* argument and returns a void*. Name your type
`pthread\_callback'

\begin{Shaded}
\begin{Highlighting}[]
\KeywordTok{typedef} \NormalTok{______________________;}
\end{Highlighting}
\end{Shaded}

\subsection{Q 1.7}\label{q-1.7}

In addition to the function arguments what else is stored on a thread's
stack?

\subsection{Q 1.8}\label{q-1.8}

Implement a version of
\texttt{char*\ strcat(char*dest,\ const\ char*src)} using only
\texttt{strcpy} \texttt{strlen} and pointer arithmetic

\begin{Shaded}
\begin{Highlighting}[]
\DataTypeTok{char}\NormalTok{* mystrcat(}\DataTypeTok{char}\NormalTok{*dest, }\DataTypeTok{const} \DataTypeTok{char}\NormalTok{*src) \{}

  \NormalTok{? Use strcpy strlen here}

  \KeywordTok{return} \NormalTok{dest;}
\NormalTok{\}}
\end{Highlighting}
\end{Shaded}

\subsection{Q 1.9}\label{q-1.9}

Implement version of size\_t strlen(const char*) using a loop and no
function calls.

\begin{Shaded}
\begin{Highlighting}[]
\NormalTok{size_t mystrlen(}\DataTypeTok{const} \DataTypeTok{char}\NormalTok{*s) \{}

\NormalTok{\}}
\end{Highlighting}
\end{Shaded}

\subsection{Q 1.10}\label{q-1.10}

Identify the three bugs in the following implementation of
\texttt{strcpy}.

\begin{Shaded}
\begin{Highlighting}[]
\DataTypeTok{char}\NormalTok{* strcpy(}\DataTypeTok{const} \DataTypeTok{char}\NormalTok{* dest, }\DataTypeTok{const} \DataTypeTok{char}\NormalTok{* src) \{}
  \KeywordTok{while}\NormalTok{(*src) \{ *dest++ = *src++; \}}
  \KeywordTok{return} \NormalTok{dest;}
\NormalTok{\}}
\end{Highlighting}
\end{Shaded}

\section{Printing}\label{printing}

\subsection{Q 2.1}\label{q-2.1}

Spot the two errors!

\begin{verbatim}
fprintf("You scored 100%");
\end{verbatim}

\section{Formatting and Printing to a
file}\label{formatting-and-printing-to-a-file}

\subsection{Q 3.1}\label{q-3.1}

Complete the following code to print to a file. Print the name, a comma
and the score to the file `result.txt'

\begin{Shaded}
\begin{Highlighting}[]
\DataTypeTok{char}\NormalTok{* name = .....;}
\DataTypeTok{int} \NormalTok{score = ......}
\NormalTok{FILE *f = fopen(}\StringTok{"result.txt"}\NormalTok{,_____);}
\KeywordTok{if}\NormalTok{(f) \{}
    \NormalTok{_____}
\NormalTok{\}}
\NormalTok{fclose(f);}
\end{Highlighting}
\end{Shaded}

\section{Printing to a string}\label{printing-to-a-string}

\subsection{Q 4.1}\label{q-4.1}

How would you print the values of variables a,mesg,val and ptr to a
string? Print a as an integer, mesg as C string, val as a double val and
ptr as a hexadecimal pointer. You may assume the mesg points to a short
C string(\textless{}50 characters). Bonus: How would you make this code
more robust or able to cope with?

\begin{Shaded}
\begin{Highlighting}[]
\DataTypeTok{char}\NormalTok{* toString(}\DataTypeTok{int} \NormalTok{a, }\DataTypeTok{char}\NormalTok{*mesg, }\DataTypeTok{double} \NormalTok{val, }\DataTypeTok{void}\NormalTok{* ptr) \{}
   \DataTypeTok{char}\NormalTok{* result = malloc( strlen(mesg) + }\DecValTok{50}\NormalTok{);}
    \NormalTok{_____}
   \KeywordTok{return} \NormalTok{result;}
\NormalTok{\}}
\end{Highlighting}
\end{Shaded}

\section{Input parsing}\label{input-parsing}

\subsection{Q 5.1}\label{q-5.1}

Why should you check the return value of sscanf and scanf? \#\# Q 5.2
Why is `gets' dangerous?

\subsection{Q 5.3}\label{q-5.3}

Write a complete program that uses \texttt{getline}. Ensure your program
has no memory leaks.

\subsection{Heap memory}\label{heap-memory}

When would you use calloc not malloc? When would realloc be useful?

(Todo - move this question to another page) What mistake did the
programmer make in the following code? Is it possible to fix it i) using
heap memory? ii) using global (static) memory?

\begin{Shaded}
\begin{Highlighting}[]
\DataTypeTok{static} \DataTypeTok{int} \NormalTok{id;}

\DataTypeTok{char}\NormalTok{* next_ticket() \{}
  \NormalTok{id ++;}
  \DataTypeTok{char} \NormalTok{result[}\DecValTok{20}\NormalTok{];}
  \NormalTok{sprintf(result,}\StringTok{"%d"}\NormalTok{,id);}
  \KeywordTok{return} \NormalTok{result;}
\NormalTok{\}}
\end{Highlighting}
\end{Shaded}

\subsection{Q1}\label{q1}

Is the following code thread-safe? Redesign the following code to be
thread-safe. Hint: A mutex is unnecessary if the message memory is
unique to each call.

\begin{Shaded}
\begin{Highlighting}[]
\DataTypeTok{static} \DataTypeTok{char} \NormalTok{message[}\DecValTok{20}\NormalTok{];}
\NormalTok{pthread_mutex_t mutex = PTHREAD_MUTEX_INITIALIZER;}

\DataTypeTok{void} \NormalTok{*format(}\DataTypeTok{int} \NormalTok{v) \{}
  \NormalTok{pthread_mutex_lock(&mutex);}
  \NormalTok{sprintf(message, }\StringTok{":%d:"} \NormalTok{,v);}
  \NormalTok{pthread_mutex_unlock(&mutex);}
  \KeywordTok{return} \NormalTok{message;}
\NormalTok{\}}
\end{Highlighting}
\end{Shaded}

\subsection{Q2}\label{q2}

Which one of the following does not cause a process to exit? * Returning
from the pthread's starting function in the last running thread. * The
original thread returning from main. * Any thread causing a segmentation
fault. * Any thread calling \texttt{exit}. * Calling
\texttt{pthread\_exit} in the main thread with other threads still
running.

\subsection{Q3}\label{q3}

Write a mathematical expression for the number of ``W'' characters that
will be printed by the following program. Assume a,b,c,d are small
positive integers. Your answer may use a `min' function that returns its
lowest valued argument.

\begin{Shaded}
\begin{Highlighting}[]
\DataTypeTok{unsigned} \DataTypeTok{int} \NormalTok{a=...,b=...,c=...,d=...;}

\DataTypeTok{void}\NormalTok{* func(}\DataTypeTok{void}\NormalTok{* ptr) \{}
  \DataTypeTok{char} \NormalTok{m = * (}\DataTypeTok{char}\NormalTok{*)ptr;}
  \KeywordTok{if}\NormalTok{(m == 'P') sem_post(s);}
  \KeywordTok{if}\NormalTok{(m == 'W') sem_wait(s);}
  \NormalTok{putchar(m);}
  \KeywordTok{return} \NormalTok{NULL;}
\NormalTok{\}}

\DataTypeTok{int} \NormalTok{main(}\DataTypeTok{int} \NormalTok{argv, }\DataTypeTok{char}\NormalTok{** argc) \{}
  \NormalTok{sem_init(s,}\DecValTok{0}\NormalTok{, a);}
  \KeywordTok{while}\NormalTok{(b--) pthread_create(&tid, NULL, func, }\StringTok{"W"}\NormalTok{); }
  \KeywordTok{while}\NormalTok{(c--) pthread_create(&tid, NULL, func, }\StringTok{"P"}\NormalTok{); }
  \KeywordTok{while}\NormalTok{(d--) pthread_create(&tid, NULL, func, }\StringTok{"W"}\NormalTok{); }
  \NormalTok{pthread_exit(NULL); }
  \CommentTok{/*Process will finish when all threads have exited */}
\NormalTok{\}}
\end{Highlighting}
\end{Shaded}

\subsection{Q4}\label{q4}

Complete the following code. The following code is supposed to print
alternating \texttt{A} and \texttt{B}. It represents two threads that
take turns to execute. Add condition variable calls to \texttt{func} so
that the waiting thread does not need to continually check the
\texttt{turn} variable. Q: Is pthread\_cond\_broadcast necessary or is
pthread\_cond\_signal sufficient?

\begin{Shaded}
\begin{Highlighting}[]
\NormalTok{pthread_cond_t cv = PTHREAD_COND_INITIALIZER;}
\NormalTok{pthread_mutex_t m = PTHREAD_MUTEX_INITIALIZER;}

\DataTypeTok{void}\NormalTok{* turn;}

\DataTypeTok{void}\NormalTok{* func(}\DataTypeTok{void}\NormalTok{* mesg) \{}
  \KeywordTok{while}\NormalTok{(}\DecValTok{1}\NormalTok{) \{}
\CommentTok{// Add mutex lock and condition variable calls ...}

    \KeywordTok{while}\NormalTok{(turn == mesg) \{ }
        \CommentTok{/* poll again ... Change me - This busy loop burns CPU time! */} 
    \NormalTok{\}}

    \CommentTok{/* Do stuff on this thread */}
    \NormalTok{puts( (}\DataTypeTok{char}\NormalTok{*) mesg);}
    \NormalTok{turn = mesg;}
    
  \NormalTok{\}}
  \KeywordTok{return} \DecValTok{0}\NormalTok{;}
\NormalTok{\}}

\DataTypeTok{int} \NormalTok{main(}\DataTypeTok{int} \NormalTok{argc, }\DataTypeTok{char}\NormalTok{** argv)\{}
  \NormalTok{pthread_t tid1;}
  \NormalTok{pthread_create(&tid1, NULL, func, }\StringTok{"A"}\NormalTok{);}
  \NormalTok{func(}\StringTok{"B"}\NormalTok{); }\CommentTok{// no need to create another thread - just use the main thread}
  \KeywordTok{return} \DecValTok{0}\NormalTok{;}
\NormalTok{\}}
\end{Highlighting}
\end{Shaded}

\subsection{Q5}\label{q5}

Identify the critical sections in the given code. Add mutex locking to
make the code thread safe. Add condition variable calls so that
\texttt{total} never becomes negative or above 1000. Instead the call
should block until it is safe to proceed. Explain why
\texttt{pthread\_cond\_broadcast} is necessary.

\begin{Shaded}
\begin{Highlighting}[]
\DataTypeTok{int} \NormalTok{total;}
\DataTypeTok{void} \NormalTok{add(}\DataTypeTok{int} \NormalTok{value) \{}
 \KeywordTok{if}\NormalTok{(value < }\DecValTok{1}\NormalTok{) }\KeywordTok{return}\NormalTok{;}
 \NormalTok{total += value;}
\NormalTok{\}}
\DataTypeTok{void} \NormalTok{sub(}\DataTypeTok{int} \NormalTok{value) \{}
 \KeywordTok{if}\NormalTok{(value < }\DecValTok{1}\NormalTok{) }\KeywordTok{return}\NormalTok{;}
 \NormalTok{total -= value;}
\NormalTok{\}}
\end{Highlighting}
\end{Shaded}

\subsection{Q6}\label{q6}

A non-threadsafe data structure has \texttt{size()} \texttt{enq} and
\texttt{deq} methods. Use condition variable and mutex lock to complete
the thread-safe, blocking versions.

\begin{Shaded}
\begin{Highlighting}[]
\DataTypeTok{void} \NormalTok{enqueue(}\DataTypeTok{void}\NormalTok{* data) \{}
  \CommentTok{// should block if the size() would become greater than 256}
  \NormalTok{enq(data);}
\NormalTok{\}}
\DataTypeTok{void}\NormalTok{* dequeue() \{}
  \CommentTok{// should block if size() is 0}
  \KeywordTok{return} \NormalTok{deq();}
\NormalTok{\}}
\end{Highlighting}
\end{Shaded}

\subsection{Q7}\label{q7}

Your startup offers path planning using latest traffic information. Your
overpaid intern has created a non-threadsafe data structure with two
functions: \texttt{shortest} (which uses but does not modify the graph)
and \texttt{set\_edge} (which modifies the graph).

\begin{Shaded}
\begin{Highlighting}[]
\NormalTok{graph_t* create_graph(}\DataTypeTok{char}\NormalTok{* filename); }\CommentTok{// called once}

\CommentTok{// returns a new heap object that is the shortest path from vertex i to j}
\NormalTok{path_t* shortest(graph_t* graph, }\DataTypeTok{int} \NormalTok{i, }\DataTypeTok{int} \NormalTok{j); }

\CommentTok{// updates edge from vertex i to j}
\DataTypeTok{void} \NormalTok{set_edge(graph_t* graph, }\DataTypeTok{int} \NormalTok{i, }\DataTypeTok{int} \NormalTok{j, }\DataTypeTok{double} \NormalTok{time); }
  
\end{Highlighting}
\end{Shaded}

For performance, multiple threads must be able to call \texttt{shortest}
at the same time but the graph can only be modified by one thread when
no threads other are executing inside \texttt{shortest} or
\texttt{set\_edge}.

Use mutex lock and condition variables to implement a reader-writer
solution. An incomplete attempt is shown below. Though this attempt is
threadsafe (thus sufficient for demo day!), it does not allow multiple
threads to calculate \texttt{shortest} path at the same time and will
not have sufficient throughput.

\begin{Shaded}
\begin{Highlighting}[]
\NormalTok{path_t* shortest_safe(graph_t* graph, }\DataTypeTok{int} \NormalTok{i, }\DataTypeTok{int} \NormalTok{j) \{}
  \NormalTok{pthread_mutex_lock(&m);}
  \NormalTok{path_t* path = shortest(graph, i, j);}
  \NormalTok{pthread_mutex_unlock(&m);}
  \KeywordTok{return} \NormalTok{path;}
\NormalTok{\}}
\DataTypeTok{void} \NormalTok{set_edge_safe(graph_t* graph, }\DataTypeTok{int} \NormalTok{i, }\DataTypeTok{int} \NormalTok{j, }\DataTypeTok{double} \NormalTok{dist) \{}
  \NormalTok{pthread_mutex_lock(&m);}
  \NormalTok{set_edge(graph, i, j, dist);}
  \NormalTok{pthread_mutex_unlock(&m);}
\NormalTok{\}}
\end{Highlighting}
\end{Shaded}

\begin{quote}
Note thread-programming synchronization problems are on a separate wiki
page. This page focuses on conceptual topics. Question numbers subject
to change
\end{quote}

\subsection{Q1}\label{q1-1}

What do each of the Coffman conditions mean? (e.g.~can you provide a
definition of each one) * Hold and wait * Circular wait * No pre-emption
* Mutual exclusion

\subsection{Q2}\label{q2-1}

Give a real life example of breaking each Coffman condition in turn. A
situation to consider: Painters, paint and paint brushes. Hold and wait
Circular wait No pre-emption Mutual exclusion

\subsection{Q3}\label{q3-1}

Identify when Dining Philosophers code causes a deadlock (or not). For
example, if you saw the following code snippet which Coffman condition
is not satisfied?

\begin{verbatim}
// Get both locks or none.
pthread_mutex_lock( a );
if( pthread_mutex_trylock( b ) ) { /*failed*/
   pthread_mutex_unlock( a );
   ...
}
\end{verbatim}

\subsection{Q4}\label{q4-1}

How many processes are blocked?

\begin{itemize}
\tightlist
\item
  P1 acquires R1
\item
  P2 acquires R2
\item
  P1 acquires R3
\item
  P2 waits for R3
\item
  P3 acquires R5
\item
  P1 acquires R4
\item
  P3 waits for R1
\item
  P4 waits for R5
\item
  P5 waits for R1
\end{itemize}

\subsection{Q5}\label{q5-1}

How many of the following statements are true for the reader-writer
problem?

\begin{itemize}
\tightlist
\item
  There can be multiple active readers
\item
  There can be multiple active writers
\item
  When there is an active writer the number of active readers must be
  zero
\item
  If there is an active reader the number of active writers must be zero
\item
  A writer must wait until the current active readers have finished
\end{itemize}

\subsection{Q1}\label{q1-2}

What are the following and what is their purpose? * Translation
Lookaside Buffer * Physical Address * Memory Management Unit * The dirty
bit

\subsection{Q2}\label{q2-2}

How do you determine how many bits are used in the page offset?

\subsection{Q3}\label{q3-2}

20 ms after a context switch the TLB contains all logical addresses used
by your numerical code which performs main memory access 100\% of the
time. What is the overhead (slowdown) of a two-level page table compared
to a single-level page table?

\subsection{Q4}\label{q4-2}

Explain why the TLB must be flushed when a context switch occurs
(i.e.~the CPU is assigned to work on a different process).

\begin{quote}
Question numbers subject to change
\end{quote}

\subsection{Q1}\label{q1-3}

Fill in the blanks to make the following program print 123456789. If
\texttt{cat} is given no arguments it simply prints its input until EOF.
Bonus: Explain why the \texttt{close} call below is necessary.

\begin{Shaded}
\begin{Highlighting}[]
\DataTypeTok{int} \NormalTok{main() \{}
  \DataTypeTok{int} \NormalTok{i = }\DecValTok{0}\NormalTok{;}
  \KeywordTok{while}\NormalTok{(++i < }\DecValTok{10}\NormalTok{) \{}
    \NormalTok{pid_t pid = fork();}
    \KeywordTok{if}\NormalTok{(pid == }\DecValTok{0}\NormalTok{) \{ }\CommentTok{/* child */}
      \DataTypeTok{char} \NormalTok{buffer[}\DecValTok{16}\NormalTok{];}
      \NormalTok{sprintf(buffer, ______,i);}
      \DataTypeTok{int} \NormalTok{fds[ ______];}
      \NormalTok{pipe( fds);}
      \NormalTok{write( fds[}\DecValTok{1}\NormalTok{], ______,______ ); }\CommentTok{// Write the buffer into the pipe}
      \NormalTok{close(  ______ );}
      \NormalTok{dup2( fds[}\DecValTok{0}\NormalTok{],  ______);}
      \NormalTok{execlp( }\StringTok{"cat"}\NormalTok{, }\StringTok{"cat"}\NormalTok{,  ______ );}
      \NormalTok{perror(}\StringTok{"exec"}\NormalTok{); exit(}\DecValTok{1}\NormalTok{);}
    \NormalTok{\}}
    \NormalTok{waitpid(pid, NULL, }\DecValTok{0}\NormalTok{);}
  \NormalTok{\}}
  \KeywordTok{return} \DecValTok{0}\NormalTok{;}
\NormalTok{\}}
\end{Highlighting}
\end{Shaded}

\subsection{Q2}\label{q2-3}

Use POSIX calls \texttt{fork} \texttt{pipe} \texttt{dup2} and
\texttt{close} to implement an autograding program. Capture the standard
output of a child process into a pipe. The child process should
\texttt{exec} the program \texttt{./test} with no additional arguments
(other than the process name). In the parent process read from the pipe:
Exit the parent process as soon as the captured output contains the !
character. Before exiting the parent process send SIGKILL to the child
process. Exit 0 if the output contained a !. Otherwise if the child
process exits causing the pipe write end to be closed, then exit with a
value of 1. Be sure to close the unused ends of the pipe in the parent
and child process

\subsection{Q3 (Advanced)}\label{q3-advanced}

This advanced challenge uses pipes to get an ``AI player'' to play
itself until the game is complete. The program \texttt{tictactoe}
accepts a line of input - the sequence of turns made so far, prints the
same sequence followed by another turn, and then exits. A turn is
specified using two characters. For example ``A1'' and ``C3'' are two
opposite corner positions. The string \texttt{B2A1A3} is a game of 3
turns/plys. A valid response is \texttt{B2A1A3C1} (the C1 response
blocks the diagonal B2 A3 threat). The output line may also include a
suffix \texttt{-I\ win} \texttt{-You\ win} \texttt{-invalid} or
\texttt{-draw} Use pipes to control the input and output of each child
process created. When the output contains a \texttt{-}, print the final
output line (the entire game sequence and the result) and exit.

\begin{quote}
Question numbers subject to change
\end{quote}

\subsection{Q1}\label{q1-4}

Write a function that uses fseek and ftell to replace the middle
character of a file with an `X'

\begin{Shaded}
\begin{Highlighting}[]
\DataTypeTok{void} \NormalTok{xout(}\DataTypeTok{char}\NormalTok{* filename) \{}
  \NormalTok{FILE *f = fopen(filename, ____ );}
  


\NormalTok{\}}
\end{Highlighting}
\end{Shaded}

\subsection{Q2}\label{q2-4}

In an \texttt{ext2} filesystem how many inodes are read from disk to
access the first byte of the file \texttt{/dir1/subdirA/notes.txt} ?
Assume the directory names and inode numbers in the root directory (but
not the inodes themselves) are already in memory.

\subsection{Q3}\label{q3-3}

In an \texttt{ext2} filesystem what is the minimum number of disk blocks
that must be read from disk to access the first byte of the file
\texttt{/dir1/subdirA/notes.txt} ? Assume the directory names and inode
numbers in the root directory and all inodes are already in memory.

\subsection{Q4}\label{q4-3}

In an \texttt{ext2} filesystem with 32 bit addresses and 4KB disk
blocks, an inodes that can store 10 direct disk block numbers. What is
the minimum file size required to require an single indirection table?
ii) a double direction table?

\subsection{Q5}\label{q5-2}

Fix the shell command \texttt{chmod} below to set the permission of a
file \texttt{secret.txt} so that the owner can read,write,and execute
permissions the group can read and everyone else has no access.

\begin{verbatim}
chmod 000 secret.txt
\end{verbatim}

\begin{itemize}
\tightlist
\item
  \href{http://angrave.github.io/SystemProgramming/networkingreviewquestions.html}{Wiki
  w/Interactive MC Questions}
\item
  See \protect\hyperlink{coding-questions}{Coding questions}
\item
  See \protect\hyperlink{short-answer-questions}{Short answer questions}
\item
  See \href{https://courses.engr.illinois.edu/cs241/mps/mp7/}{MP
  Wearables} Food For Thought questions
\end{itemize}

\hypertarget{short-answer-questions}{\section{Short answer
questions}\label{short-answer-questions}}

\subsection{Q1}\label{q1-5}

What is a socket?

\subsection{Q2}\label{q2-5}

What is special about listening on port 1000 vs port 2000?

\begin{itemize}
\tightlist
\item
  Port 2000 is twice as slow as port 1000
\item
  Port 2000 is twice as fast as port 1000
\item
  Port 1000 requires root privileges
\item
  Nothing
\end{itemize}

\subsection{Q3}\label{q3-4}

Describe one significant difference between IPv4 and IPv6

\subsection{Q4}\label{q4-4}

When and why would you use ntohs?

\subsection{Q5}\label{q5-3}

If a host address is 32 bits which IP scheme am I most likely using? 128
bits?

\subsection{Q6}\label{q6-1}

Which common network protocol is packet based and may not successfully
deliver the data?

\subsection{Q7}\label{q7-1}

Which common protocol is stream-based and will resend data if packets
are lost?

\subsection{Q8}\label{q8}

What is the SYN ACK ACK-SYN handshake?

\subsection{Q9}\label{q9}

Which one of the following is NOT a feature of TCP? - Packet re-ordering
- Flow control - Packet re-tranmission - Simple error detection -
Encryption

\subsection{Q10}\label{q10}

What protocol uses sequence numbers? What is their initial value? And
why?

\subsection{Q11}\label{q11}

What are the minimum network calls are required to build a TCP server?
What is their correct order?

\subsection{Q12}\label{q12}

What are the minimum network calls are required to build a TCP client?
What is their correct order?

\subsection{Q13}\label{q13}

When would you call bind on a TCP client?

\subsection{Q14}\label{q14}

What is the purpose of socket bind listen accept ?

\subsection{Q15}\label{q15}

Which of the above calls can block, waiting for a new client to connect?

\subsection{Q16}\label{q16}

What is DNS? What does it do for you? Which of the CS241 network calls
will use it for you?

\subsection{Q17}\label{q17}

For getaddrinfo, how do you specify a server socket?

\subsection{Q18}\label{q18}

Why may getaddrinfo generate network packets?

\subsection{Q19}\label{q19}

Which network call specifies the size of the allowed backlog?

\subsection{Q20}\label{q20}

Which network call returns a new file descriptor?

\subsection{Q21}\label{q21}

When are passive sockets used?

\subsection{Q22}\label{q22}

When is epoll a better choice than select? When is select a better
choice than epoll?

\subsection{Q23}\label{q23}

Will \texttt{write(fd,\ data,\ 5000)} always send 5000 bytes of data?
When can it fail?

\subsection{Q24}\label{q24}

How does Network Address Translation (NAT) work?

\subsection{Q25}\label{q25}

@MCQ Assuming a network has a 20ms Transmit Time between Client and
Server, how much time would it take to establish a TCP Connection? 20 ms
40 ms 100 ms 60 ms @ANS 3 Way Handshake @EXP @END

\subsection{Q26}\label{q26}

What are some of the differences between HTTP 1.0 and HTTP 1.1? How many
ms will it take to transmit 3 files from server to client if the network
has a 20ms transmit time? How does the time taken differ between HTTP
1.0 and HTTP 1.1?

\hypertarget{coding-questions}{\section{Coding
questions}\label{coding-questions}}

\subsection{Q 2.1}\label{q-2.1-1}

Writing to a network socket may not send all of the bytes and may be
interrupted due to a signal. Check the return value of \texttt{write} to
implement \texttt{write\_all} that will repeatedly call \texttt{write}
with any remaining data. If \texttt{write} returns -1 then immediately
return -1 unless the \texttt{errno} is \texttt{EINTR} - in which case
repeat the last \texttt{write} attempt. You will need to use pointer
arithmetic.

\begin{Shaded}
\begin{Highlighting}[]
\CommentTok{// Returns -1 if write fails (unless EINTR in which case it recalls write}
\CommentTok{// Repeated calls write until all of the buffer is written.}
\NormalTok{ssize_t write_all(}\DataTypeTok{int} \NormalTok{fd, }\DataTypeTok{const} \DataTypeTok{char} \NormalTok{*buf, size_t nbyte) \{}
  \NormalTok{ssize_t nb = write(fd, buf, nbyte);}
  \KeywordTok{return} \NormalTok{nb;}
\NormalTok{\}}
\end{Highlighting}
\end{Shaded}

\subsection{Q 2.2}\label{q-2.2}

Implement a multithreaded TCP server that listens on port 2000. Each
thread should read 128 bytes from the client file descriptor and echo it
back to the client, before closing the connection and ending the thread.

\subsection{Q 2.3}\label{q-2.3}

Implement a UDP server that listens on port 2000. Reserve a buffer of
200 bytes. Listen for an arriving packet. Valid packets are 200 bytes or
less and start with four bytes 0x65 0x66 0x67 0x68. Ignore invalid
packets. For valid packets add the value of the fifth byte as an
unsigned value to a running total and print the total so far. If the
running total is greater than 255 then exit.

\subsection{Give the names of two signals that are normally generated by
the
kernel}\label{give-the-names-of-two-signals-that-are-normally-generated-by-the-kernel}

\subsection{Give the name of a signal that can not be caught by a
signal}\label{give-the-name-of-a-signal-that-can-not-be-caught-by-a-signal}

\subsection{Why is it unsafe to call any function (something that it is
not signal handler safe) in a signal
handler?}\label{why-is-it-unsafe-to-call-any-function-something-that-it-is-not-signal-handler-safe-in-a-signal-handler}

Coding Questions

Write brief code that uses SIGACTION and a SIGNALSET to create a SIGALRM
handler.
